The recent technological development driven by digital technologies and IT solutions has led to the progressive digitalization of our societies and to radical changes, notably in customer preference. This changes created new business opportunities and software startups are acknowledged to be fast in exploiting them \citep{paternoster2014software, seppanen2016initial}. As a consequence,

here defined following \citet{blank2005four, ries2011lean, unterkalmsteiner2016software} as \textit{organizations designed to search for a repeatable, profitable and scalable business model under uncertain condition, time constraints, lack of resources, using cutting-edge tools and depending on technological waves of software industry"}

\footnote{Various complementary conceptions of human capital having an impact on firm performance have been proposed by scholars over the past decades. First, the \textit{"general"} conception of human capital, that does not directly relate to venture tasks (for e.g., formal education such as economic, scientific managerial, etc. and professional experiences such as technical, commercial, etc.). Second, the \textit{"specific"} conception of human capital that refers to knowledge related to peculiar markets and industry specific contexts. Third,the \textit{"task-related"} conception of human capital, which refers to the current task of the venture (for e.g., start-up experience, industry experience, business skills) \citep{gibbons2004task}. Finally, another differentiation was suggested by entrepreneurship scholars regarding human capital investment (i.e., formal education and professional experience) and outcomes of human capital investment (i.e., the skills, knowledge, and abilities that derive from education and experience) \citep{unger2011human, marvel2016human}. All this conceptions build on solid theoretical basis such as human capital theory \citep{becker1964human} or resource based view of the firm \citep{wernerfelt1984resource, barney1991firm}}.

Several decades of reasearch into human aspects of entrepreneurship and firms confirmed that human capital fuel firms growth jointly to other factors. However, not all human capital compositions necessarily match the needs of the different growth phases a startup may encounter, in particular investors-backed software startups because they cope with high uncertainty and follow a non-lineary growth.

Based on the assumption of the resource-based view of the firm, HC variables are considered as unique and distinctive resources endowments for firms value creation and performance over time \citep{penrose1956theory, becker1964human}. As such, demographic characteristics of individuals (e.g. age, sex, race, ethnicity) as well as HC investment (proxied by e.g., educational background, startup or industry previous experiences) and HC outcomes (e.g. knowledge, skills and abilities), be specifics or not, are considered as decisive components jointly to environmental factors, to explain ventures likelihood of achieving performance (proxied by e.g., firm size, growth, profitability, job creation or fundraising). In this literature, empirical studies often aggregate individual HC at the firm level, considering the HC of certain individuals as the sum of the HC of the whole team \citep{marvel2016human}.
Since \citet{cavallo2019fostering}, : Delmar (2006) pointed out that the growing demand is the first stage of new ventures' growth process, which it will consequently lead to higher sales and to hire new employee. Thus, digital new ventures with high growth in sales or employee may be realistically considered as firms that are successfully embracing and overcoming the scaling phase
In this line, Software-as-a-Service (SaaS) business model startups belongs to the software-based ventures branch that deliver applications over the Internet (cloud computing) associated with subscription-based revenue logic in contrast to software licencing business \citet{luoma2018exploring}.
In this line, Software-as-a-Service (SaaS) business model startups belongs to the software-based ventures branch that deliver applications over the Internet (cloud computing) associated with subscription-based revenue logic in contrast to software licencing business \citet{luoma2018exploring}.

Since \citet{cavallo2019fostering}, : Delmar (2006) pointed out that the growing demand is the first stage of new ventures' growth process, which it will consequently lead to higher sales and to hire new employee. Thus, digital new ventures with high growth in sales or employee may be realistically considered as firms that are successfully embracing and overcoming the scaling phase
- What is Human Capital ?\\

footnote{The question \textit{"how HC teams influence performance over time?} is well documented in the literature, and is nurtured by two complementary theories: 1/ Resource bases theory (Penrose) (classic theory) and 2/ Competence based theory (follow resource-based theory). The latest focus on competence movement and is developed among others by Hamel/Prahalad (1994), Sanchez et al. (1996), Teece et al. (1997), Freiling (2004). It offers theory of sustaining competitive advantage and a quite dominant framework in strategic management (i.e. Bresser et al. 2000; Barney 2002)}

For example, in a longitudinal study focusing on founders and TMT (not on full team members) group composition and turnover of teams over time, \citet{beckman2007early} argue that previous work experiences affiliations is an important predictors of the ability to attract VC funding or complete an IPO. On their side, \citet{reese2020should} argue that the sharedness of entrepreneurial competencies is positively related to new ventures' total amount of funding acquired, while sharedness of managerial skills is not.

Does it exist any common patterns of team characteristics among software startup teams?

Since \citep{de2010interrelationships}, Human Capital for employees are \textit{Employees’ human capital refers to the unique set of knowledge, skills and abilities of workers acquired from education and experience (Becker 1964). It reflects a large part of the stock of knowledge within an organisation (Grant 1997; Cabrera and Cabrera 2003). In line with previous work, we opt for employees’ educational level as a proxy for their human capital. Hambrick and Mason (1984), for example, argue that employees’ formal education reflects their knowledge bases and cognitive abilities. According to Smith et al. (2005), education helps individuals to improve their understanding of what they know, to more accurately predict outcomes, to better manage time and resources and to monitor results. In addition, education provides new explicit information and knowledge that greatly influence an individual’s cognitive reasoning skills.}

Human capital are idiosyncrasic to every individuals and provide (1) access to networks (2) increase problem-solving attitude and behavior (3) create better strategies (4) provide management skills (5). Since \citep{de2010interrelationships}, \textit{Rauch, Frese and Utsch (2005), employees with higher levels of education have higher intellectual potential to learn and accumulate general knowledge (Hitt et al. 2001) as well as firm-specific skills and knowledge (D’Aveni 1996).}\newline

Since \citep{marvel2016human}, \textit{Human capital has been argued to be of higher utility when it applies to the specific task that needs to be performed. For example, the transfer of education and experience works best if old and new activities share common situation–response elements. Thus, it may be helpful to distinguish between constructs that are task related and non-task related (Cooper et al., 1994). Task-related human capital includes those types of human capital that relate to the current task of the venture (e.g., start-up experience, industry experience, business skills). Conversely, non-task-related human capital includes types of human capital that do not directly relate to venture tasks (e.g., formal education, employment experience).}\newline

Since \citep{marvel2016human}, \textit{Human capital has been argued to be of higher utility when it applies to the specific task that needs to be performed. For example, the transfer of education and experience works best if old and new activities share common situation–response elements. Thus, it may be helpful to distinguish between constructs that are task related and non-task related (Cooper et al., 1994). Task-related human capital includes those types of human capital that relate to the current task of the venture (e.g., start-up experience, industry experience, business skills). Conversely, non-task-related human capital includes types of human capital that do not directly relate to venture tasks (e.g., formal education, employment experience).}\newline

Since Zarutskie (2010), task-specific human capital are viewed as accumulated experience related to specific tasks of importance for VC performance.

Since \citep{marvel2016human}, this distinction is mentioned in the future lines of research : they need a more fine grained approach and it is more relevant \textit{Becker (1964) theorized that knowledge and skills are the result of investments in education and work experience. Thus, most studies have used education or experience to assess human capital. However, these represent investments in human capital rather than fully realized knowledge and/or skills (i.e., outcomes). Past research has provided evidence that outcome-based human capital constructs are better direct indicators of human capital, whereas investment-based indicators are viewed as indirect predictors of human capital (Davidsson, 2004). For example, Unger et al. (2011) suggest the entrepreneurship–success relationship is higher for outcomes of human capital than for investments alone because investments are indirect indicators and thus one step removed. While some entrepreneurs may have the same education or highly similar work experience, the readily available knowledge or skills possessed may be dramatically different (Keith & Frese, 2005).}\newline

Regarding OUTCOMES of Human capital, you have 3 of them \citep{marvel2016human} : KNOWLEDGE ; SKILLS ; ABILITIES. Researchers pointed out that knowledge/skill/abilities are contextual and different among individual in a singular situation bring different outcome for the firm. As such, HC (knowledge, skill or abilities - distinguished in their paper) is crucial when specific. \newline

Since \citep{marvel2016human}, skills, knowledge, and abilities:
\begin{itemize}
  \item : Knowledge \textit{is the possession and understanding of principles, facts, processes, and the interactions among them. Knowledge tends to be of greater value when it is specific to a particular domain and when related to specific entrepreneurial activities (Markman & Baron, 2003). Enterprising individuals or firms must have knowledge, especially of the market and any relevant technology that is critical to success. Knowledge can range from generic to specific areas in terms of task, job, organization, or industry. It is usually clustered within domains such as those learned through formal education (e.g., accounting, marketing, information systems, electrical engineering).}. Also, \textit{Shane et. al 2000 demonstrated how knowledge of customer problems, markets, and ways to serve markets impacts the discovery of opportunities. In a related study, Dimov (2007) illustrated how domains of market and technology knowledge impact the development of opportunities. Such knowledge outcomes can be gained through investments in education, training, experience, or the recruitment of key individuals.}
  \item : Skills \textit{are also human capital outcomes but refer to observable applications or know-how. Skills are not necessarily enduring characteristics and depend on experience or practice. These are usually task specific or closely related to a set of tasks. For example, Heneman, Judge, and Kammeyer-Mueller (2009) identified varying job-related skills that range from basic (e.g., public speaking, mathematics, active learning) to cross-functional (e.g., social skills, problem-solving skills, technical skills). Skills that apply specifically to an entrepreneurial task may provide advantages within the entrepreneurship process. A variety of skills can be developed through investments in training or experience and can also be developed in combination with education and practice. Of particular interest is how skills specific to the entrepreneurial process can be developed.}
  \item : Abilities \textit{is the third human capital outcome and is an underlying or enduring characteristic useful to performing a range of tasks. At the individual level, ability is often associated with general traits such as the ability to reason inductively. Abilities differ from skills in that they are less likely to change over time and they are applicable across a wide set of tasks that may be encountered in many different contexts (Nyberg, Moliterno, Hale, & Lepak, 2014). While abilities cannot be developed in the same manner as knowledge or skills, entrepreneurs and firms can acquire abilities via investments in team members, alliances, and organizations.}
\end{itemize}

Since \citet{beckman2007early}, we should mention demographical characteristics instead of human capital characteristics as it is more accurate to our study : demographical characteristics CONTAINS human capital characteristics. As mentioned : \textit{Demographic arguments are importantly distinct from human capital arguments in that demography theorizes about team composition and diversity in addition to the existence of any particular experience.}

Since \citet{beckman2007early}, \textit{Among the scholars who apply organizational demographic theories to entrepreneurial settings, most have focused on a group dynamics interpretation of heterogeneity hypothesizing (but then often failing to find) negative consequences of diversity (e.g. Ensley et al., 2002; Watson et al., 2003; Chowdhury, 2005; Chandler et al., 2005).} Indeed, researchers opened a debate : Homogeneity ans Diversity. From \citep{steffens2012birds}, \textit{Previous research identifies 2 types of team homogeneity as important drivers of team performance : demographic and HC.} On one side, some suggest that team members homogeineity improve communication and reduce team conflict (watson 1993, Ancona and Caldwell 1992). On the other side, others say that diverse team are more effective (complementary skills) Meakin and Snaith 1997. More, in \citet{taylor2006superman}, \textit{It has been argued that communication in large groups has a process cost that reduces group out-puts (Kurtzberg & Amabile, 2001; Steiner, 1972).}\newline

By extension regional contexts, \citep{audretsch2015entrepreneurship} underline that previous research highlighted the role of human capital in regional growth and development (Chinitz 1961) and supports a positive relationship (Glaeser et al. 1992; Audretsch 1995; Rodriguez-Pose and Crescenzi 2008).

Since \citet{andries2014small}, It is widely accepted that an organization’s capability to perform is closely tied to its intellectual capital, i.e. to its ability to utilize its individual knowledge resources --> HC. Here we have to prove the link between individual knowledge aggregated to firm knowledge and firm performance.

Since \citet{colombo2005founders}, the difference between a venture that work and another than do not work can be explained at 2 levels. First level is about initial funding. Second is Knowledge (Funding or knowledge gap). As such, we don't take into account Funding and try to understand the impact in very KNOWLEDGE intensive (startups cost = 0, we don't need so much money on the internet now).

From : \citep{marvel2016human}, \textit{First, human capital is vital to discovering and creating entrepreneurial opportunity (Alvarez & Barney, 2007; Marvel, 2013). Human capital also aids in exploiting opportunities by acquiring financial resources and launching ventures (Bruns, Holland, Shepherd, & Wiklund, 2008; Dimov, 2010). Third, human capital assists in the accumulation of new knowledge and the creation of advantages for new firms (Bradley, McMullen, Artz, & Simiyu, 2012; Corbett, Neck, & DeTienne, 2007). In practical application, human capital is the most frequently used selection criteria among venture capitalists when evaluating potential venture per-formance (Zacharakis & Meyer, 2000).}\newline

However :
- Past studies are fragmented : focus either on TMT or employees, not a full set of individuals belonging to ventures\newline
- Samples studied vary between articles (small /medium sized, etc.)\newline
- Few studies focus on digital startup \newline
- To our knowledge, there is no studies focusing specifically on software startup.\newline
- Choice of outcome is also fragmented : innovation, growth, etc.
-	very few studies within this stream have integrated human capital, entrepreneurship, and additional economic theories. EX : Signal / Agency / HC  “Agency theory (Eisenhardt, 1989) and transaction cost economics (Williamson, 1991) help us to understand how entrepreneurs without resources can marshal the means to launch ventures.”\\
- Improvement in fine grained approach : \textit{most research has relied on very coarse measures, and there is a clear need for finer-grained approaches that reflect more precise variance among aspects of human capital. For example, investments in education are commonly operationalized by years of education or completion of a university degree. While this operationalization is a way to leverage archival data, there is clear room for improvement.} see \citet{marvel2016human}

TMT or upper echelon : Since \citet{andries2014small}, \textit{On the one hand, a number of articles adhering to the ‘upper echelon perspective’ study the link between CEOs’ and top managers’ human capital and small firms’ innovative performance. The results are mixed. While some studies (e.g. Chaganti et al., 2008) find a positive relationship, others observe no relationship (e.g. Lynskey, 2004).}. TMT theory lies in the upper echelon theory provided by Hambrick and Mason (1984) which udnerline the importance of founders in new ventures. The assumption is, since \citep{cooper1994initial}, that for new or small businesses \textit{“the firm is built around the entrepreneur}. Since \citet{beckman2007early}, TMT is studied a lot (40 empirical studies) and we have to distinguich 2 kind of arguments : demographical variables (composition and diversity + HC), and Human capital variables (such as prior experiences). \textit{the past two decades has seen an explosion in research exploring how the composition or demography of groups may affect group process and performance. As Williams and O’Reilly (1998) report, over 40 empirical studies have examined the effects of TMT demography on firm outcomes, team dynamics and firm. Entrepreneurial research has begun to address how the compositional characteristics of founding teams affect performance (Amason et al., 2006; Eisenhardt and Schoonhoven, 1990; Chandler et al., 2005; Chowdhury, 2005; Roure and Maidique, 1986; Watson et al., 2003). However, most scholars draw upon human capital theories and study characteristics such as the type and amount of prior experience present on a team (Aldrich and Zimmer, 1986; Cooper et al., 1994; Gimeno et al., 1997; Schefczyk and Gerpott, 2000; Burton et al., 2002; Baum and Silverman, 2004). Demographic arguments are importantly distinct from human capital arguments in that demography theorizes about team composition and diversity in addition to the existence of any particular experience.}\\

Best introduction to this topic from \citet{beckman2007early}, \textit{Founding teams are important because founding team characteristics impact organizational structure and performance (Kimberly, 1979; Boeker, 1988; Eisenhardt and Schoonhoven, 1990; Baron et al., 1996). Furthermore, reports in the business press as well as academic research note that venture capital firms pay particular attention to the strengths and weaknesses of founding team when deciding whether to fund a new venture (Goslin and Barge, 1986; Heileman, 1997; Baum and Silverman, 2004). Yet, it is also often suggested that as entrepreneurial firms evolve and mature they need to attract people with different skills (Aldrich, 1999; Boeker and Karichalil, 2002). As firms grow and age, founders are likely to be replaced with experienced executives, particularly in VC backed firms (Hellmann and Puri, 2002). Thus, entrepreneurial teams are likely, as a matter of course, to experience compositional changes as new executive team members are added and as others are replaced. The importance of particular characteristics may change, and thus we examine both teams over time. Importantly, instead of relying on aggregate measures of tenure heterogeneity as is often the case, with a longitudinal research design it is possible to study both entrances and exits (Ucbasaran et al., 2003) and assess their relative impact (Chandler et al., 2005). Doing so begins to identify whether it is more advantageous to begin with a large, functionally diverse team or to add people and capabilities over time. We examine a traditional team demographic characteristic, functional heterogeneity, as well as a new kind ofdemographic characteristic — background affiliation — that may be particularly relevant for the success of young firms. Managers bring a good deal of tacit knowledge with them from their prior firms, about how to organize and manage work processes, and this knowledge is likely to differ even between two firms in a similar industry. We examine these characteristics for founding teams and subsequent TMTs and estimate their impact on achieving important firm outcomes.} p5 --> We need to adapt this with a full-employees approach + TMT + Founders

Example of results of TMT / founders perspective.

\begin{itemize}
  \item TMT / founders : from \citep{de2010interrelationships}, \textit{In line with upper echelon theory (Hambrick and Mason 1984), innovation in start-ups is associated with owners/managers’ human capital. First, their formal education and experience in other organisations determine the unique set of skills or knowledge base that owners/managers bring to the organisation (Boeker 1997). Second, highly educated owners/managers are more receptive to new ideas (Hambrick andMason 1984). Finally, prior experience plays a prominent role in successful opportunity recognition (Hills and Shrader 1998; Shane 2000). A unique set of knowledge and skills, receptivity to new ideas and opportunity recognition skills are all necessary ingredients for innovation. Prior research has provided evidence of this relationship in start-ups (Lynskey 2004)}.\\

  \item TMT / Founders : from \citet{delmar2006does}, there is a huge spot for our study : litterature need fine-grained study AND longitudinal. \textit{SPOT FOR OUR STUDY : our measures were relatively coarse-grained. However, we believe that the methodological limitations overcome in this study suggest the importance of future studies of the human capital of new venture founding teams that employ more fine-grained measures, using similarlongitudinal designs on representative samples of new firms measured from their initiation.}

  \item TMT / founders : from \citep{gruber2012minds}, \textit{there is a fairly large body of literature investigating the relationship between founding team composition and the evolution and performance of start-up firms and, more generally, the relationship between team composition and a variety of outcomes in established firms (e.g., Kor, 2003; for a recent overview see Finkelstein, Hambrick, & Cannella, 2009). This literature indicates that teams with diverse backgrounds provide a heterogeneous set of knowledge, cognitive abilities, skills, and information, which when combined provide a broader knowledge base from which to solve organizational problems. Team diversity can be assessed along different demographic dimensions such as education or organizational tenure (Taylor & Greve, 2006).}\\

  \item Founders : from \citet{beckman2007early}, \textit{Roure and Keeley’s (1990) : study of new ventures that reported team bcompletenessQ (the degree to which key positions were staffed by members of the founding team—was associated with firm success. Having broad functional experience represented on the team also makes a firm more attractive to external stakeholders and to investors. It signals that the management team has the requisite skills and capabilities to make the firm successful, profitable, and thereby a worthwhile investment).}

  \item TMT / Founders : from \citep{unger2011human}, \textit{Accordingly, human capital theory was originally developed to explain variations in financial returns of employees. Applied to entrepreneurship this means that entrepreneurs strive to receive financial returns from their venturing activities relative to their human capital investments. Therefore, entrepreneurs' human capital should be positively associated with a preference for venture scale and growth (Cassar, 2006).}\\

  \item TMT / Founders / longitudinal study : \citep{beckman2007early} is a longitudinal study. They focus as us on group composition and turnover on teams over time. The only thing that they did not study is educational background and full teams. \textit{we go beyond human capital explanations of teams and also consider group composition and turnover on teams as important predictors of new venture success.}

  \item TMT / founders limits : since \citep{de2010interrelationships}, studies on HC of TMT \textit{are limited, somestimes contradictory and inconsistent - (see Webber and Donahue 2001 + Lester et al. 2006 for some critics)}). Indeed, they mentioned that we have no answer yet regarding the relation of TMT and Innovation performance. (HC education is fine, HC work xp and innovation is inconsistent). \textit{A consistent finding in research is a positive relationship between the level of education of owners/managers and the receptivity to new ideas and innovation (e.g., Smith et al. 2005). Regarding industry experience findings are less consistent. Prior industry experience may play a prominent role in successful opportunity recognition processes (Hills and Shrader 1998; Shane 2000). On the other hand, experienced owners/managers are vulnerable since they may unwittingly stick to manners of working which are commonly accepted in the industry and are less able to grasp new ideas (Hambrick and Mason 1984; Ruef 2002). In general, we expect that the owners/managers’ educational level will be positively related with innovation. Because of mixed prior results, we do not formulate a hypothesis on the effect of owners/managers’ industry experience on innovation.}\\

  \item TMT / founders limits : since \citep{marvel2016human}, studies that focus on founders aggregate founders HC and treat this as an extention of the whole firm, which is not the case. \textit{The high level of consistency in operationalization is also one of the primary concerns related to human capital research. We find that firm-level human capital has been treated as an extension of individual-level human capital. Most firm-level research within this stream has assumed that human capital within the firm is a direct function of individual human capital—such as the founder. Some studies assessing firm-level human capital have used similar, if not the exact same, measures as those assessing individual-level human capital and then summed the presence of that individual level to represent that of the organization (e.g., Cassar, 2006; Eddleston, Kellermanns, & Zellweger, 2012). These approaches leave room for innovative measures that could more accurately represent the aggregate of the firm or recognize the potential for synergies, such as combinative effects of human capital.}\\

  \item TMT : since \citet{chandler2005antecedents}, \textit{Previous research suggests that changes in the size and composition ofa management team may influence the development ofthe business (Boeker, 1997).}

  \item Founders : since \citet{reese2020should}, which is a VERY similar stdy but focus only on founders \textit{Human capital helps founders address the many difficult tasks a new venture has to accomplish, such as creating new knowledge, identifying and exploiting opportunities, and acquiring financial means (Jin et al., 2017). Accordingly, a plethora of empirical studies find, at least in most cases, positive associa-tions between founders' human capital and various performance measures (Marvel, Davis, & Sproul, 2016; Unger, Rauch, Frese, & Rosenbusch, 2011). While many initial investigations focus on the one founder, more recent applications of the human capital theory in entrepreneurship acknowledge that entrepreneurial initiatives are often the efforts of a founding team (Honore, 2015; Marvel et al., 2016), so some researchers claim that today's entrepreneur is likely to be plural (Colombo & Grilli, 2005), an observation which is in line with the birth of many prominent ventures, including Google, Apple, and PayPal. Most empirical applications, while also finding positive associations with firm perfor-mance, treat founding teams' human capital as the sum of individual team members' human capital (Baum, Locke, & Smith, 2001; Ko & McKelvie, 2018; Marvel et al., 2016).}\\
\end{itemize}

%_________________________________________ Founders, TMT + employees : integration of knowledge of ALL individuals in the equation}

About employees from individual to the organisation : employees matters in organisations since they are impotant contributors to venture growth. As such, measuring the evolution of firms \\

Why focus into individuals ? Since \citet{andries2014small}, most studies focus on Founders / TMT = upper echelon. We need a integrated approach we claim that employees are usefull for attracting VC. \textit{While the ‘upper echelon’ literature finds mixed evidence regarding the link between innovative firms’ CEOs’ and top managers’ human capital and small for innovation the role of performance, employees non-managerial performance has only been studied in large firms. /!\ => OUTPUT = INNOVATION (We look for VC) Conclusion : the historical focus on the entrepreneur/CEO which was broadened more recently to the study of entrepreneurial teams does not yet fully capture small firms’ innovative potential.}

Since \citet{andries2014small}, individuals HC are unique resources of knowledge that provide competitive advantage for the firm. \textit{A critical portion of this knowledge required for innovation resides with individuals (Subramaniam and Youndt, 2005; Hansen, 1999; Szulanski, 1996). Individuals are the primary agents of knowledge creation and in the case of tacit In addition, individual knowledge, the principal repositories of knowledge (Grant, 1997). knowledge and skills (like other intangible resources such as brand equity) are more likely to product a competitive advantage because they are often rare and socially complex, thereby making them difficult to imitate (Hitt et al., 2001). As a result, human capital is regarded as a critical resource in developing innovations and a sustainable competitive advantage (Hitt et al., 2001; Barney, 1991; 1995; Lado and Wilson, 1994; Barney and Wright, 1998; Huselid, 1995; Wright and McMahan, 1992; Wiig, 1997)I.}

Example of results of TMT / founders + employees perspective.

\begin{itemize}
  \item TMT / Founders / Employees and Innovation : Since \citep{de2010interrelationships}, there is a link between innovative outputs of the firm and HC of employees (and TMT) : \textit{Smith et al. (2005) studied employees’ human capital, they concluded that the number of years of education and the diversity in employees’ knowledge bases have a positive impact on the knowledge creating capability of large technology firms. This knowledge creating capability subsequently influences the rate of new products and services. Conversely, Subramaniam and Youndt (2005) found a negative impact of employees’ human capital on radical innovative capability in a firm with 100 employees or more. This relationship, however, was moderated by social capital, implying that human capital only incites innovation if it is networked, shared and channeled through relationships.}. A huge critics to this paper (Smith et al. 2005) is that they take only Key Knowledge WOrkers : which is in fine TMT. Another Critic to Subramaniam and Youndt (2005) is that they send a questionnaire to executive : it is not clear what they asked and to who (executives, employees ?)\\

  \item Employees : since \citep{de2010interrelationships}, there is the mix of skills of employees and employees by themselves that bring value to the company \textit{The combination of new and existing knowledge incites a learning process, the creation of fresh insights and the discovery of new opportunities. Moreover, through contacts with independent board members or experts the owners/managers’ social network expands. This may, under certain circumstances, enhance the capacity for creative action and innovation (Ruef 2002). Second, although employees are few in numbers in start-ups, owners/managers do mention them as necessary resources (Bergmann Lichtenstein and Brush 2001). They are carriers of tacit knowledge and carry out the productive and innovative work of the firm. Therefore, we assume that innovative output in start-ups also depends on the human capital, i.e., the knowledge and competences, of employees.} Finally, the results provided by De Winne and sells, 2010 : find a indirect relationship between CEO and empployees knowledge. Since \cite{andries2014small}, \textit{De Winne and Sels (2010) finds that owners’ and managers’ human capital has no direct effect on new ventures’ innovative output, but only an indirect effect in the sense that highly educated CEOs and managers tend to hire more highly educated employees and tend to use more human resource practices (see also Bergmann Lichtenstein and Brush, 2001; Borch et al., 1999; Schuler and Jackson, 1987);}\\

  \item Employees : since \citep{de2010interrelationships}, the next sentence is related to H2 of De Winne study is usefull to argument on the importance to take into account ALL employees in startups : \textit{Compared with existing and larger companies, newly formed firms usually work in less planned and formalized ways and cannot afford to establish separate R&D departments. The development, acquisition or transformation of new knowledge in start-ups thus heavily depends on all employees. Consequently, the way in which employees’ human capital is developed over time and the manner in which workers are managed are extremely important in newly established firms (Hayton 2003; Ciavarella 2003).} As a resumee, besides the fact that it mentioned the HRM practices, the author state that the achievement of business objectives requires a skilled, valuable and unique workforce over time. Indeed : \textit{They do so to attract and retain employees that are necessary for optimal and continuous functioning of the firm. Bacon and Hoque (2005) have tested this statement in small firms and concluded that the workforce skill-mix is indeed one of the most important determinants of the presence of HRM in small firms. Small firms employing highly skilled workers are more likely to introduce HR practices.}\\

  \item Employees Empirical : From \citep{steffens2012birds}, this study use longitudinal data of entrepreneurial teams. This is great however, they collected data for the respondent up to five team members --> we don't know the whole population of tema members. Furthermore, we don't know about Task-relatedness and job-relatedness.

  \item Employees Empirical :"A longitudinal study of success and failure among scientist-started ventures" --> longitudinalstudy

  \item : MUST CITED - work on turnover in new ventures (Boeker and Karichalil, 2002; Chandler et al., 2005) But are TMT oriented. + \citet{milosevic2016venture} \textit{While early research studies focused on individual human capital (Cooper et al., 1994), recent stu-dies suggest the importance of the human capital characteristics of entire teams for firm success (Amason et al., 2006; Gruber et al., 2012).}

  \item Employees limits : since \citep{marvel2016human}, however, studies on employees failed to bring the structuring of human capital characteristics to a fine enough granularity to sharply capture competencies needs variations throughout startups lifespan see for example : \citep{davila2003venture, alemany2005unbiased, engel2007firm, steffens2012birds}.\\
\end{itemize}

throughout firm life cycle. Indeed, \citet{seppanen2016initial} mentioned that \textit{the availability of correct competencies is a crucial success factor in software startups. Studying competency needs is, however, not a straightforward task. Startups can greatly differ, are needed in parallel, and different many competencies competencies are needed in different phases of a startup's evolution.}.
However, competencies needs depend on life cycle as \citet{chandler2005antecedents} mentioned : \textit{2/ demands on a team may differ at different developmental stages (cf. Birley and Stockley, 2000). Possible differences in requisite team characteristics at different developmental stages have been noted in the evolutionary literature (cf. Aldrich, 1999), but such speculations have not been verified empirically in the literature on entrepreneurial teams.}
As a matter of fact, GOSS competencies needs differs greatly from classic organization growth or established software companies and they rely on external source of financing, as business angels or venture capitalists, to foster their unpredictable and non-linear growth \citep{unterkalmsteiner2016software, huang2017growing, nambisan2017digital, cavallo2019fostering}.

%_________________________________________ Resumee of the results regarding CEO+TMT+employees HC impact on firm performance (innovation / VC / survival)}

####### EXCEL ########
See Delmar 2006 for a first resumee on TMT
See Storey 1994 Understanding the Small Business Sector - for a second resumee --> \citet{colombo2005founders}, \textit{In a survey ofempirical studies on the determinants of small firms’ growth, Storey (1994) highlights that only 8 studies out of the 17 surveyed find a strong positive effect ofentrepreneurs’ education.} TMT / Founders
See Aboramadan 2020 for a review of TMT literature (great columns segmentation)

We are here NOT in micro-organizational, but macro-organizational as we don't focus on relationships between workers,how to manage the startup team, but  \textit{e study how organizations move in markets, and how their strategies regarding employees and leadership affect the performance of the entire organization.}
https://bizfluent.com/13304253/how-to-manage-different-ab-c-personality-types-at-work

\subsection{Growth-oriented-software-startups funding and life cycle }

Questions like \textit{Why do firms collapse and what are the driving forces?} or \textit{How startups teams change over time and perform ?} are well documented in the literature by two kind of theories. Indeed, there are 2 competing kind of theories we can use (see Freiling, Jörg - Article A competence-based theory of the firm) for the distinction :
- Resource bases theory (Penrose) (BASIC - ORIGIN)
- Competence based theory (FOLLOW R-B-Theory)(competence movement, developed among others by Hamel/Prahalad (1994), Sanchez et al. (1996), Teece et al. (1997), offers undoubtedly a promising theory of sustaining competitive advantage and a quite dominant framework in strategic management (i.e. Bresser et al. 2000; Barney 2002))
wx

From \citet{chandler2005antecedents} \textit{2/ demands on a team may differ at different developmental stages (cf. Birley and Stockley, 2000). Possible differences in requisite team characteristics at different developmental stages have been noted in the evolutionary literature (cf. Aldrich, 1999), but such speculations have not been verified empirically in the literature on entrepreneurial teams.}

From \citep{siepel2017non} and \citet{hanks1994tightening}, the firm aging literature is very dense. \textit{The firm aging literature (Henderson, 1999; Sorensen and Stuart 2000; Thornhill and Amit 2003) demonstrates the range of challenges that firms face as they progress from new firms, into ‘adolescence’ (see Aspelund et al. (2005), Courderoy et al. (2012) on new technology based firms, or NTBFs), and on into maturity. As firms age they face different requirements for human capital, investment (financial capital) and market challenges.} However, the main author that gather many litterature on this topic is \citet{hanks1994tightening}. He mention that we can call : life-cycle stages / growth stages / developmental stages, depend of the authors. Hnaks provide a definition that we can follow : \textit{life-cycle stage as a unique con-figuration of variables related to organization context and structure. This notion is further supported by Galbraith (1982) who used the term reconfiguration in character-izing the transition from one stage to the next.}\\

Since \citet{cavallo2019fostering}, phases of digital growth and life cycle \textit{According to recent contributions, digital startups typically are com-mitted to testing and validating their business model, while digital scaleups already show significant traction on customers, a validated business model, and they have been funded through a first Series ‘A’ round – over 1 million dollar (Autio, 2016). Proving a constant growing demand or a higher interest in the value proposition from the customers is commonly perceived by venture ca-pitalists as signal of “traction” and (more importantly) as signal of a proven business model. The matter is so crucial that the answer pro-vided by entrepreneurs will determine the very success of a fundraising process. A twofold reason testifies the complexity and fuzziness char-acterizing the concept of traction. First, metrics of traction may vary significantly (also within the digital sector) according to the specific product–market addressed by the firm (Onetti, 2014). Second, scholars argue that user base (Prasad et al., 2010) rather than customer base is the heart of the rapid scaling (Huang et al., 2017). Delmar (2006) pointed out that the growing demand is the first stage of new ventures' growth process, which it will consequently lead to higher sales and to hire new employee. Thus, digital new ventures with high growth in sales or employee may be realistically considered as firms that are successfully embracing and overcoming the scaling phase. Recently, when referring to ‘successful’ scaleups, scholars also used the term gazelles (Duruflé et al., 2017) as to indicate firms that are rapidly growing (e.g.Birch et al., 1993; Henrekson and Johansson, 2010). Even though there is no complete agreement among scholars (March and Sutton, 1997), the most widely used threshold which defines “gazelles” (Birch et al., 1993; Henrekson and Johansson, 2010) is having a yearly sales growth rate of at least 20/100 for three or more consecutive years (Fischer and Reuber, 2011; Sims and O'Regan, 2006). In this study, we will focus particularly on the earlier and thus most critical growth stages: startup and scaleup. In particular, based on the literature before mentioned, the letter will include only those new digital firms entering the scaling phase (by receiving a first Series ‘A’ round), not yet con-siderable as gazelles.}

Life cycle of startups. Since Hanks et al. 1994, in a paper called Tightening the life-cycle construct: a taxonomic study of growth stage configurations in high-technology organizations., aging firm need for different skills, priorities, and structural configurations. They propose a taxonomy of firm life cycles. We should mention it and adapt it for SaaS Startup as follow.

Since \citet{cavallo2019fostering}, there are multiple stages of venture lifetime \textit{Organizations grow through multiple stages over their lifetime. This process has been along debated in the literature for about fifty to sixty years (e.g. Fisher et al., 2016; Gaibraith, 1982; Mintzberg, 1984; Penrose, 1959; Phelps et al., 2007) providing several models, both linear (e.g. Greiner, 1972) and non–linear (e.g. Orser et al., 2000). The literature mainly deals with rapidly growing hi–tech new ventures (e.g. Phelps et al., 2007). As regards, an early and remarkable contribution in this direction and specifically focused on technology–based ventures comes from Gaibraith (1982). This population of ventures captures higher attention due to the extreme pressures they experience while growing and their supreme need for adaptation and change (Phelps et al., 2007). Technology (including digital technology) has been treated in entrepreneurship literature as a context for empirical work (e.g. Cavallo et al., 2018a; Colombo and Grilli, 2005; Hahn et al., 2018). Today, an ongoing debate is recognizing a superior and pervasive role of digital technologies into entrepreneurial process and outcome (Nambisan, 2017), and, thus, in digital new venture growth. Digital Technologies are a powerful tool deeply influencing digital new venture growth by creating affordances to quickly scale (Autio et al., 2018).}

Since \citet{hanks1994tightening} : Stages

\begin{itemize}
  \item - Stage 1 = \textit{a start-up stage of development} = young, small firms. The mean age is just over four years,annual sales average near 271,000, and mean employment is 6.46 employees. On the average, these firms are growing quite rapidly, with sales growing at 91/100 and employ-ment at 29/100. The basis of organization structure is simple (1.21), with a mean of 2.2 organization levels. The organization is highly centralized (19.29) and quite informal (Formalization = 38.92). Firms at this stage employ a mean of 1.5 specialized func-tions. The most consistently present specialty is research and development, present in 86/100 of the firms in this cluster. As no other specialized functions are present in more than 25/100 of the firms, it is apparent that the focal priority appears to be product development.
  ----> The firms are relatively young, small, highly centralized and informal, focusing on the development and early commercialization of their technology-based product(s).
        In practical terms, venture capitalist base their SEED funding decisions primarelly on the entrepreneur (the jockey), irrespective of the service or product (the horse), the market and the spatio-temporal contexts in which ventures emerge and develop (horse race) and the initial fundings preconditions (odds) \citep{macmillan1985criteria}.
  Since \citep{cooper1994initial}, \textit{resources are fundamental at the beginning of a firm lifespan, whatever the environment do because firm with stronger resources will perform the environment.} \citep{cooper1994initial}, .
  \item - Stage 2 : \textit{expansion stage of development} = The mean age is 7.36 years, mean employment is 23.64 employees, and mean sales are approximately 1.4 million. Relative to other stages, firms at this stage average the highest rate of sales and employment growth. Sales growth averages at 297/100 and employment is growing at 94/100 . Firms in this stage have generally adopted a functional basis of organization (Struc-ture = 2.00). Organization decision making is still very centralized (18.08), but less so than in Stage I, and organization systems are a little more formal than in Stage I (45.88). Compared to Stage I firms, they have an additional organization level (Levels = 3.18) and 3.4 specialized functions (Specialization = 4.91). In addition to research and development, specialized functions present in at least 50/100  of firms include sales and accounting, indicating that these firms are actively involved in the commercialization of their product(s).
  \item - Stage 3 : \textit{later expansion/early maturity stage of development.} = While on the average firms in Cluster C are slightly younger (mean Age = 6.66) than Cluster B, mean size is more than twice as large. Stage III firms employ a mean of 62.76 employees and have average annual sales of over 3.7 million. Firms in this stage are still growing quite rapidly, but not quite as fast as firms in Stage II. Mean sales growth is 99/100 and mean employment growth is 28/100. Companies in this group average four levels of management, and generally employ a functional organization structure. These firms have the lowest centralization mean of all the clusters (14.45), and the second highest level of formalization (mean = 52.89). Firms at this stage employ a mean of 10.17 specialized functions. Additional full-time specialists present in Stage III, not present in Stages I or II, include shipping and receiving, finance, purchasing, quality control, customer/product service, production planning and scheduling, and payroll. This appears to indicate expansion and increased professionalization, particularly in the manufacturing arm of the firm.
  \item - Stage 4 : \textit{between the Maturity and Diversification stages of development.}Firms in Cluster 0 averaged 16.2 years of age, employed a mean of 495 employees, and averaged just under 46 million in annual sales. These companies experienced sales growth of 37/100 and employee growth of near 57/100. They average 5.7 organization levels and 15.3 specialized functions. While the majority of firms employ a functional orga-nization structure, a divisional structure has emerged in several (Structure = 2.4). Centralization is low (15.10), and formalization is the highest of all the clusters (53.2), though just slightly higher than Stage III. Specialized functions present in this config-uration, over and above those present in the preceding stages, include personnel. build-ing maintenance, advertising, market research, and inventory control. Presence of these specialists may suggest greater formalization of human resource programs and policies, cost control, and market expansion.

For other source of life cycle org : The Organizational Life Cycle: Review and Future Agenda. Also, we can mention here : (Boeker & Karichalil, 2002 - entrepreneurial transition).

As such, these life-cycle can be affiliated to Explorative / Exploitative phases of firms. Indded, since \citet{grillitsch2020does} \textit{An alternative theoretical explanation for the negative time profile in growth lies in the notion of changes in the venture's strategic focus over time. Life-cycle models have typically argued that new ventures cycle through different phases.}
\begin{itemize}
    \item Explorative since \citet{grillitsch2020does} : \textit{During the inception phase, ventures typically have not yet explored their entrepreneurial opportunity to its full extent. They are often not fully informed about the core technology underlying their products or services, the demand side, or the main competitors (Alvarez et al., 2013; Knight, 1921). In this respect, exploration activities increase the information on the potential customer value of a specific entrepreneurial opportunity (Alvarez et al., 2013) and there-by reduce the uncertainty over whether the firm's offerings will be able to compete on the market(Choi & Shepherd, 2004). Thus, during their inception phase, firms are typically more focused on exploring the characteristics of their entrepreneurial opportunity than on exploiting an opportunity, which might be premature and lock the firm into a (potentially bad) local optimum on a rugged land-scape (Levinthal, 1997).}
    \item Exploitative since \citet{grillitsch2020does} \textit{As the ventures reduce technological and demand uncertainty through explo-ration, they become better informed about the type and value of their entrepreneurial opportunities, which allows them to develop strategies to exploit them effectively. If uncertainty is sufficiently low, firms shift their focus to exploit their entrepreneurial opportunity, which is based on tasks that aim at preserving patterns of action which have proven to be goal-efficient (Becker, 2005).}
\end{itemize}

As such, there is a selection mechanisms in teams along Stages. note that stage 1 doesn't have any money incentive, and the other, yes.

Distinction between startup and scalup --> emergence of a new model by scholars and practicioners. \citet{cavallo2019fostering}, \textit{Concerning digital new venture growth, an emergent model is gaining attention among both scholars and practitioners (Blank, 2013; Isenberg and Onyemah, 2016; Huang et al., 2017; Autio et al., 2018; Srinivasan and Venkatraman, 2018). The model applies particularly to digital entrepreneurship and is based on a fundamental distinction between startups still working on validating their business model and scaleups showing significant metrics of traction on customers, and being already funded through a first Series ‘A’ round – over 1 million dollars (Autio, 2016). This distinction in the early phase of growth between startups and scaleups is worth studying, to enhance our understanding on digital entrepreneurship (Srinivasan and Venkatraman, 2018). Hence, recent research developments open to further contributions on digital new ventures growth process and the role of emerging venture financing trends involving angel groups and VC funds.} = we know that VC impact startup growth. However, we don't know on digital technologies.

\subsection{Linking human capital, ventures growth and funding}

Avenues of research pointed out by \citet{steffens2012birds} indicate the direction to follow \textit{"does the composition of new venture team by relational demographics and HC change over time ? And if so, in what direction ?"}\\

Predicting new venture performance is so difficult Since A paper called "The Automated Venture Capitalist: Data and Methods to Predict the Fate of Startup Ventures} --> \textit{More broadly, teams “in the wild” tend not to formally leverage the half a century of research in this area when creating companies, academic collaborations, or other ventures. One ambitious and recent attempt to formally leverage the existing knowledge was made by Google’s Project Aristotle (PA). The project investigated 180 teams within Google over several years in an effort to improve team formation and cohesion within the organization. Even with unprecedented levels of information at their disposal (from team gender balance and educational background to lunch habits), the project’s results were mostly inconclusive (Duhigg 2016). Such results highlight that the prediction of team performance, even for the most technically capable, is a problem without an easy solution. The difficulty of predicting team performance may explain why even the most quantitatively inclined invest-ment firms still use in-person meetings and “gut-feel” to evaluate investment candidates, with perceptions of team passion and trustworthiness often overriding quantitative metrics of performance (Sudek 2006); even then, rates of identification do not exceed 25/100 (Wiltbank et al. 2009; Gage 2012).}

We combine : Life cycle + exploration / exploitation : to make our Hypothesis.
Regarding VC explaining growth, 2 main authors : davila and gellman and puri (profesionalisation of entrepre). moreover, since \citet{cavallo2019fostering}, \textit{Relevant contributions argue how external funding may add value to portfolio companies by providing certification effect when coming from venture capitalist (e.g. Dimov and Shepherd, 2005; Puri and Zarutskie, 2012) as well as angels (e.g. Carpentier and Suret, 2015).} + Nowadays, the equity funding landscape displays a great variety of sources (Drover et al., 2017). --> must be cited for Angel / VC / crowdfunding + accelerator funding + \textit{According to Carpentier and Suret (2015), a professionalization and for-malization of the angel market (historically considered as informal VC market) is taking place, which is leading angels to act and impact si-milarly to VCs. Also, VC funds demonstrate a growing interest in earlier stages than in the past (Dutta and Folta, 2016). Those new trends in-volving both angels and VC funds may be perceived as an answer to the higher level of uncertainty (as well as new opportunities) brought in by digital technologies, which are leading to an increased unpredictability and non–linearity of new ventures growth (Huang et al., 2017; Nambisan, 2017).}

Aspects of HC influence the transition from stage 1 to stage, 2, but the ones in Stage 2 may have a different impact on stage 3. Indeed, it is key here to have a sequence / timing analysis. From \citep{marvel2016human}, \textit{Aspects of human capital influence the transition from one stage of the entrepreneurial process to another. However, a particular type of human capital may be essential to accomplishing a milestone, while the same human capital may be less important, or even disadvantageous, to other milestones within the process. For example, using conceptual-izations of general versus specific human capital, research has shown that specific human capital investments are beneficial to nascent entrepreneurs and venture development— whereas general human capital investments did not have an effect (Davidsson & Honig, 2003). Conversely, other research has shown that general types of human capital invest-ment are helpful for the achievement of initial public offerings, but specific human capital was of little value (Dimov & Shepherd, 2005). This evidence suggests that indeed the effects of human capital are of unequal value when considering different phases, or milestones, along the entrepreneurial process. It is quite plausible that human capital central to explaining one phase of the process may have little influence on later stages in the process. Another possibility is that although the vast majority of studies assume more human capital is universally better, some findings suggest that aspects of human capital can also hinder venture milestones such as opportunity discovery and product innovation (Marvel, 2013; Marvel & Lumpkin, 2007). This illustrates the need for more carefully constructed studies that fully investigate dimensions of human capital specific to mile-stones along the process. This research stream can also benefit from considering contex-tual conditions in which human capital is applied to a particular phase or milestone.}. As a consequence, the factors contributing to performance in startups evolve over time. The challenges also (even if they are path-dependent). Indeed, In line with signal theory and Ko and McKelvie : \textit{we argue that not all signals will be equally important in different funding stages due to the different challenges a new venture faces at various stages.}.

%_________________________________________ OWN MONEY INCENTIVE

For SaaS Startups, the process is the following :


- \citet{crowne2002software} : 3 steps
\begin{itemize}
  \item STARTUP phase : All software product companies start with an entrepreneur and a vision. They see a market opportunity and know how to exploit technology to satisfy it. They need to assemble a small executive team around them with the necessary skills, then start to build the product. This may require additional people to be recruited into development, but cash is usually in short supply.

  ### DEVELOPERS ARE INEXPERIENCED
  --> inexperienced people : inexperienced developers neglect non-coding issues such as architecture, design, testing, configuration management, deployment and documentation
  --> The principal developer for the company must be highly experienced, and familiar with all aspects of software engineering practice
  --> product development must be experienced team leaders to provide a solid foundation for subsequent growth.

  ### PRODUCT ISN’T REALLY A PRODUCT : SEED IS NEEDED
  --> company wanting to evolve a product from a custom solution should expect to spend the same amount of cash as originally expended to develop it, several times over
  --> Additional cash will he needed to transform the company from a service to a product organization, developing capabilities such as product support, professional services, marketing and sales

  ### PRODUCT HAS NO OWNER
  --> impossibility of building an outstanding product unless its evolution is controlled by an individual or small group of like- minded people [2]. Initially the product is often specified by a group of developers working closely with a visionary entrepreneur.
  --> The owner may be a market-oriented engineer or a development-oriented marketer who is responsible for communicating between development, sales and marketing
  --> The product owner must develop this themselves through constant contact with customers and the market place.
  NO STRATEGIC PLAN FOR PRODUCT DEVELOPMENT
  --> It is a team-oriented activity where new people can take months to become productive.
  --> Ensure that there is a strategic plan [4] for the company, with clear objectives in the short and medium term [5].

  ### PRODUCT PLATFORM IS UNRECOGNISED
  --> The importance of technologies and components which form part of the product is not understood, discussed and managed. Selection of these components is left entirely to product developers.

  \item STABILIZATION : By now the entrepreneur typically needs more cash and must approach extemal investors

  ### FOUNDERS WON'TLET GO
  --> There is damaging conflict between new executives and the founders of the company who may also be major shareholders. People continue to look to the founders for product and thought leadership, although new leaders have been appointed to provide these.
  --> A ship cannot have two captains. If a clear and accountable executive role cannot be identified for each founder, and they are unwilling to become subordinates, then they are likely to play a disruptive role as the company evolves. \textit{role of founders well delimited and complementary enhance firm ability to raise}
  --> Founders must either assume a mainstream executive role, truly accept a subordinate position, or join the board as a nonexecutive director

  ### DEVELOPMENT TEAM FAILS TO GEL
  --> There is a divide between developers who join the company early and those who are recruited later. The early developers demand special treatment which is not justified by their experience and skills. Early developers mount significant resistance to organizational change.
  --> This is a result of recruiting too many inexperienced developers during the startup phase. To compensate, experienced people will be brought into senior positions \textit{the good mix between devleopers : junior and senior enhance firm ability to raise + more management hierarchi (lead dev)}
  --> Promote early developers who show technical and leadership potential. Rapidly eliminate any weak links or other hiring mistakes.

  ### PRODUCT IS UNRELIABLE
  --> only the best developers can solve them. Development of new features is slowed or halted. +This is usually caused either by inexperienced developers or a product that is released too soon. Inexperienced developers introduce more defects into the product and fix them more slowly, if at all.

  ### REQUIREMENTS BECOME UNMANAGEABLE
  --> Diverse product requirements will emerge as the sales team sells almost any?hmg to almost anyone to keep the company atloat. The product owner needs to support them while controlling new features carefully to maintain the conceptual integrity of the product [2] and not overwhelm the development team
  --> A business process is required to capture new requirements, prioritize them, then assess their feasibility and value. This process is managed by the product owner \textit{product owner is key at this moment}

  ### PRODUCT EXPECTATIONSARE TOO HIGH
  --> raising money and developement team is an issue : size, skills etc.
  --> This depends on the skill and size of the development team, and the time available to them
  --> it is impoltant for investors to check the experience and capability of the development team
  --> product must include include product features and non-functional requirements such as security, reliability, scalability and performance. \textit{some key skills must be here : scalability, perf, reliability, etc. in the dev team}

  ### SERVICE PROVISION DELAYS DEVELOPMENT
  --> Selling and commissioning the product requires skills only available in the development team. Such services include technical presales implementation, product integration and customer support

  \item GROWTH : all business processes necessary to support product development and sales are in place.

  ### SKILL SHORTAGE DEIAYS DEVELOPMENT
  --> Product development depends on a small number of skilled individuals who become a bottleneck in all activities
  --> Individuals who have survived in the company from the early days develop a wide range of skills on an ad-hoc basis
  --> Identify important skills and the people who possesses them. \textit{here a hypothesis is : what are the important skills for serie A?}

  ### PIATFORM CREEP DELAYS DEVELOPMENT
  --> Establish a business case for additional platform components before including them in the development plan

  ### PRODUCT PIPEUNE IS EMPlY
  --> many ideas for new products, but no effective way to decide
  --> To achieve this the company must invent new product ideas aligned with its strategic plans and which can be realized from the existing product platform and skills base. Whilst
  --> Commit resources to the invention and development of new products. Proposals should include projected revenues and costs for product development and introduction. Each

  ### NO PROCESS FOR PRODUCTINTRODUCTION
  --> Teams fail to assign people to work on new product introduction.
  --> As the company grows, new product introduction requires a coordinated program of activities across functional areas including product development, professional services, support, sales and marketing. This program should be run by the product owner, supported if necessary by a program manager.

  \item MATURITY : Product development contains a diverse team of multi-skilled individuals, perhaps including specialists in design, development, testing, configuration management, quality assurance, documentation and user interface design. Each person understands how their activities support the company's strategy and has personal objectives connected to that strategy

Stage 1 : gestation / early / liability of newness (Stichombe)\\
\begin{itemize}
  \item Founders and early stages : since \citep{de2010interrelationships}  \textit{In the first stage of a firm’s life cycle, owners/managers are the central actors in conceiving the company’s strategy. Subsequently, their human capital is the main source of knowledge needed to select the appropriate resources and effectively build and use the firm’s capabilities (Dutta, Bergen, Levy, Ritson and Zbarack 2002; Reinmoeller 2004). In accordance with employees’ human capital, the human capital of owners/managers is derived from their education and past experience. It is their formal education, exposure to and experience in other organisations that determines the unique set of skills or knowledge base that they bring to the new organisation (Boeker 1997).}
  \item Founders and early stages : \citep{steffens2012birds}, \textit{generally, heterogeneous teams perform better in higly uncertain environment (Ensley et al. 2002)} This means that for software startups in tech industry, this argument should be taken alongside the lifespan of the startup.(homogeneity Vs. time)
  \item Founders characteristics and VC funding : from \citep{delmar2006does}, \textit{ Shane and Stuart (2002) have looked at the effect that founding team characteristics have on the ability of new ventures to raise external capital.}
  \item TEAM and early stages : \cite{beckman2007early}, \textit{Our results confirm that team composition in the early years has dramatic effects on the firm’s ability to reach milestones, and we see the benefits of diversity and change.}
  \item Since experience in IoT Valley, when investors bid on a startup project when seed / Series A, they bid rather on experiences and people (the team), rather than the project :  they expect pivot and switch in projects.
  \item Article from Lependeven : https://business.lesechos.fr/entrepreneurs/financer-sa-creation/0604109830934-les-fondateurs-element-essentiel-de-la-valorisation-d-une-start-up-340472.php#xtor=EPR-21-[entrepreneurs]-20201030
  \item Life cycle and turnover are linked and important for new ventures team. : since \citet{chandler2005antecedents}, \textit{studies suggest that initial team size and composition are related to the future development of the new venture in complex ways, and that the adding or dropping of members from a new venture team may influence strategic direction and performance.}
  \item team composition at the begenning : \citet{chandler2005antecedents}, \textit{The most compelling research finding to date is that team-founded ventures appear to achieve better performance than individually founded ventures (Cooper and Bruno, 1977; Teach et al., 1986; Weinzimmer, 1997). Team start-ups perform better than individual start-ups; however, a larger initial team size is likely to be associated with greater heterogeneity, which can lead to increased levels of conflict due to diverging perspectives and viewpoints (Amason and Sapienza, 1997).}
  \item at the begenning, does matter diversity in knowledge or deep knowledge ? since \citet{taylor2006superman}, for creative work (Cf. Variance perf Vs. Mean perf - Task-related), \textit{For individuals, combining diverse experiences does not have the coordination or access problems that arise in teams, so an individual can have more integrated, diverse knowledge without the interper-sonal conflicts present in teams. As a result, an individual creator is less likely to make compro-mises in the creative process. Although negative extremes might occur, they will be due to the in-trinsic risks of experimentation and not to difficul-ties in communicating or reaching agreement. As a result, individual knowledge combination should give even greater performance variation than that of teams.}\\
  \item since \citet{reese2020should}, founders at the very beginning and until the 5 years matters : \textit{context of new ventures (i.e., ventures founded no more than 5 years ago that must strug-gle with the challenges that arise from their “liability of newness”; Henderson, 1999; Stinchcombe, 1965). These new ventures typically have few resources, few established internal norms with respect to appropriate behaviors, and few established relationships with external stakeholders (e.g., customers) (Kazanjian, 1988). Therefore, the founders have a high degree of influence as they direct their new ventures through the entrepreneurial process to establish a legitimate entity that can compete in the marketplace (Bruns, Holland, Shepherd, & Wiklund, 2008; Jin et al., 2017). It is not surprising, then, that their human capital plays a major role as they deal with the challenges of leading and developing the new venture (Bhagavatula, Elfring, van Tilburg, & Bunt, 2010; Davidsson & Honig, 2003; Ensley & Hmieleski, 2005). Given the general resource scarcity a new venture faces (Zhang, Soh, & Wong, 2009), the founders' human capital determines to a large degree the new venture's available resources and how well it deals with the}
  \item since \citep{cavallo2019fostering}, this bootsrapping part for software startups : \textit{In-itially, new ventures will be financed from ‘internal’ sources i.e. foun-ders' or family and friends' savings or by generating some form of in-come with early product developments. This is the so called ‘Bootstrapping’, a common practice especially among software en-trepreneurs – which acts as an alternative or complementary source of capital in the very early new venture stages (Freear et al., 1994; Freear et al., 2002).}
\end{itemize}

%_________________________________________ LOW MONEY INCENTIVE

Stage 2 : seed
\begin{itemize}
    \item since \citep{cavallo2019fostering}, the next step involve funding from external sources : \textit{since Subsequently, private individuals or ‘angels’ operating outside formal financial institutions are willing to invest their own re-sources in new business, originating an ‘informal’ venture capital market (Wetzel, 1983). When the new ventures are able to show some form of track record, they have built a proven business model and are in need of a larger amount of funding, then it is time to approach a formal financial institution such as independent venture capital fund and corporate venture capital fund (Hellmann and Thiele, 2015).}
\end{itemize}

%_________________________________________ STRONG MONEY INCENTIVE

Stage 3 : Series A / commercialisation\\
\begin{itemize}
  \item Regarding the recruitment we tend to see homophily when team size increase. Since \citep{de2010interrelationships} : \textit{in line with the similarity attraction paradigm (Byrne 1971) and the attraction-selection-attrition model (Schneider 1987), it is likely that the owner/manager recruits employees that are similar to him/herself in terms of human capital (Garcia, Posthuma and Colella 2008).}
  \item --> second round of funding, challenges change : from \citep{ko2018signaling} \textit{investors pay attention to will also vary over time. This shifting can be seen as a reflection of investors' requirement that founders achieve certain milestones as a criterion for future investment (Pahnke et al., 2015).} -->  \textit{In the second round of funding, we argue that the main signal source switches from the founder's potential ability to validate market demand or commercialize a new product to signals of the firm's ability to achieve its growth potential via innovation and market reach (Ruhnka and Young, 1991). This reasoning aligns with a growth logic in the second funding stage as investors pay closer attention to the value of potential economic returns rather than the uncertainty of the venture's ability to commercialize (a factor that is a more critical in the first round of funding). Therefore, in the second funding stage, investors tend to focus on ventures' growth signals related to scalability and innovativeness.}
  \item In developed organizations - such as Series A or B, TMT turnover matters : \citet{chandler2005antecedents}, \textit{In developed organizations, heterogeneity has been found to be directly related to management team turnover (Harrison et al., 1998; Jackson et al., 1991; Wiersema and Bird, 1993)by reducing social integration, communication, and cohesion (Wagner et al., 1984).}
\end{itemize}

Stage 4 : Series B, C / Scale up : a startup that found its business/market fit on multiple clients and want to replicate it without changing the organizationnal process and the product and so on.\\
\begin{itemize}
  \item From \citep{delmar2006does} about venture age in later stages there is a decreasing or increasing return. \textit{founding team experience might show increasing or decreasing returns, and might vary with organizational age (Brüderl and Schussler, 1990; Fichman and Levinthal, 1991), prior studies exploring the effect of founding team experience on new venture performance posit a linear relationship that does not vary with venture age or with the amount of founding team experience.}
  \item From \citep{delmar2006does} : "experience effects face diminishing marginal returns i.e.avoir un xp dans une entreprise fait ++++ sur la courbe d'apprentissage, tandis que la 3èm entreprise, l'xp fait seulement ++ par exemple ==> rendement marginal décroissants. --> CF Cyert and March, 1963/
  La meme démarche est vrai pour l'age de l'entreprise et l'xp que l'on y apporte. /// Same for Aging :  \textit{new ventures change as they get older (Churchill and Lewis, 1983). In particular, their activities become more and more complex, making the per-formance of new ventures less directly linked to the characteristics of the found-ing team and more the result of other factors. In addition, as ventures age, labor begins to be divided among organization members and the proportion of labor accounted for by the founder decreases (Witt, 2000). Consequently, founder-specific characteristics influence a smaller portion of venture activities as ventures age}
\end{itemize}

%_________________________________________ graph showing the difference between lifecycles of normal VS. startup software.

 what kind of competencies get into software startups and at what stage of their lifespan?

 To what extend software startup funding influence workforce structuration and competency composition, and vice-versa?


First set of hypothesis is linked to firm lifespan / lifecycle. As such, employees composition (mix of skills) develop and transform as the firm is aging a

Second set of Hypothesis is linked to HC.

)\\

1er set d'hypohtèse HC mixity :\newline

From \citep{marvel2016human} since figure 3 --> Task specific (Outcomes) = HC differs in life cycle and in mixity (diversity). The best approach is : Marvel  2016 or "Superman or the fantastic four? Knowledge combination and experience in innovative teams (Taylor and Greve, 2006)In this paper, they provide a good incentive about : exploration (looking for product market fit - (March, 1991)) and explotation (maintain and refines current activities)\newline
As such, \citet{taylor2006superman} make a difference between diverse knowledge of multiple domain and deep knowledge in a specific domain affect MEAN perf OR VARIANCE perf.

Also, we should differenciate : Education VS knowledge /// Training and experience VS Skills
--> Sales / Devs = on a construit le produit, on switch / proéminence du sales puis rétention ! Cf Lifen.\newline
H1a and b : generic or specific at the begenning / after Series A\newline
H2a and b : novice or experts at the begenning / after Series A\newline
Ici, il est possibe de poser plusieurs hypothèses selon le timing des boites.\newline

Rendements marginaux des founders (cf. Delmar 2006) mais se différencier et l'appliquer sur toute l'équipe : voir s'il peut s'appliquer sur un gorupe plus gros. (HC and Aging too) Donc au niveau de l'éducation, et au niveau de l'industry.\newline
As such here we can take into account that learning is subject to decay and obsolescence. What we learned from the university is not anymore useful 10 years after : the latest experiences are more relevent for people. It is mentioned in \citet{taylor2006superman} : \textit{Learning by doing is subject to decay and obsolescence (Argote et al., 1990), so it is most effective when recent produc-tion is high. It follows that a work unit with a full workload is likely to perform better than one that is underemployed.}

Adding new skills depend on diversity of founders ?
Since \citet{grillitsch2020does}, \textit{Furthermore, future research may zoom in on the type of new skill recruited to the firm conditional to the existing portfolio of skills in the firm. For instance, do integra-tion costs differ when adding a manufacturing skill to a founding team of natural scientists as com-pared to adding a social science skill to a founding team trained in forestry? Related to this is also the question of whether the importance of timing for adding new skills depends on the diversity of found-ing teams. Such more nuanced analyses could support the development of typologies with high rele-vance for management practice.}

2ème set d'hypothèse : Turnover and funding :\newline

As mentioned by \citet{chandler2005antecedents}, there is an hypothesis very linked to the one we want to introduce that say that aging firms have turnover linked to their aging. However, their conclusion doesnt talk about skills that go in / that go out. \textit{As firms move through various stages of growth, the problems that must be addressed change. This results in a need for different skills, priorities, and structural configurations (Hanks et al., 1993; Kazanjian and Rao, 1999). Both contingency (Lawrence and Lorsch, 1967) and institutional theorists (DiMaggio and Powell, 1983; Meyer and Rowan, 1977) argue that organizations change their structures to match environmental contextual demands. This suggests that as a firm develops, it puts increasing pressure on the management team to change. This change can be at least partially addressed by adding and dropping team members. Hypothesis 2. The stage of development of a company is positively associated with turnover (both additions and departures) in new venture teams.}

In this same study, \citet{chandler2005antecedents} mentioned firm Stage of development developed by Hanks et al. (1993). This study \textit{showed that over 95/100 of the variance existing between firms at different stages of development is accounted for by size in the number of employees and business age.}

Furthermore, this study mentioned a DIFFFERENCE between TEAM DEPARTURES and TEAM ADDITIONS.
In this regards, team departures : \textit{The literature on teams suggests that changes in the team can require substantial socialization (Adelman and Frey, 1997) and can impact team performance. Indeed, case study evidence suggests that changes in team membership may substantially influence the development of a venture (Chandler and Lyon, 2001).The literature on personnel turnover provides some guidelines regarding the performance implications of turnover. Starting with departures, scholars agree that both functional and dysfunctional departures exist at both the individual and organizational levels of analysis (Shaw et al., 1998). According to Dess and Shaw (2001) a cost–benefit perspective argues that recruiting, hiring, and decreased productivity costs result in diminished financial returns. From a human capital perspective, departing members may take valuable tacit and explicit knowledge away with them (Cascio, 1999). Both of these perspectives suggest a negative relationship between departures and performance. However, there is substantial empirical evidence that, on balance, the poorest performers are likely to leave (McEvoy and Cascio, 1987). A rationale explaining this finding is that both employers and employees recognize when things are not working out well. The combination of not achieving and not being valued in the organization creates an environment in which poor performers are more likely to leave than good performers. As poor performers leave, there are anticipated positive implications for team performance.} On the contrary, team additions : \textit{The dynamics associated with adding team members are different from those associated with dropping team members. Newly formed groups are more likely to operate informally, with tacit cohesion rather than following explicit rules and goals (Polanyi, 1967). Thus, even in the initial stages of a team, adding members can be disruptive, but the disruption of adding a member increases as the team and organization develops. Adding members into an existing team requires considerable socialization and adaptation by both new and existing team members (Adelman and Frey, 1997). Both new and existing group members must be clear about both what their expected goals are, and what their roles are in accomplishing various tasks (Kemer et al., 1985). In addition, sufficient time must be allowed to functionally integrate all members of a team (Heinen and Jacobson, 1976). We believe these findings apply to the specialized case of new venture teams. Existing team members develop shared expectations for how the business should operate. New members often arrive with very different perspectives and agendas. They require time to learn the cultural norms established by existing new venture members. As a result, individuals added to the team sometime after start-up are very disruptive to existing practices and are expected to be largely dysfunctional.}

Another linked hypothesis is the U-shaped inversed cited by \citet{taylor2006superman} : \textit{It has been argued that communication in large groups has a process cost that reduces group out-puts (Kurtzberg & Amabile, 2001; Steiner, 1972). An integration of the influences from a broader set of inputs and more costly communication in large teams would seem to predict a curvilinear or even an inverted-U-shaped relation between team size and innovativeness. This proposition is most rele-vant for [big] teams} --> The primary effect of task conflict is to reduce team performance (De Dreu & Weingart, 2003) --> more knowledge and diverse profiles ==> greater potential of team conflict (Williams & O Reilly, 1998).

Why does turnover exist in companies ? Consequence of VC and growth

\begin{itemize}
  \item Reason #1 : Internal stimuli \citet{chandler2005antecedents} turnover in TMT is linked to internal and external stimulis \textit{Evidence suggests that both forces external to the team and team characteristics influence member change in the team. For example, Boeker (1997) suggests that changes in the top management team are a response to changing external environmental factors} Indeed, regarding this argument on environment external factors, there is a link. \citet{beckman2007early} state that \textit{a contingency argument where it is in turbulent environments (and entrepreneurial environments are often turbulent) that there are benefits for turnover because firms need to learn and adapt to change. Entrances are an important part of that ability to learn and adapt.}

  \item Reason #2 : External stimuli : \citet{chandler2005antecedents} turnover in TMT is linked to internal and external stimulis \textit{Evidence suggests that both forces external to the team and team characteristics influence member change in the team. For example, Boeker (1997) suggests that changes in the top management team are a response to changing external environmental factors.}

  \item Reason #3 : VC investment. Turnover in firms do not only consider TMT, but also funders. See factors influencing founder departure (Boeker 2002). Also, since \citet{beckman2007early}, \textit{we find a counterintuitive result for the founder exit: it increases the rate of going public. This may be because it is those founders that exit are the poorest performers (Chandler et al., 2005). In examining the data in more detail, it appears that firms with venture capital backing are those that benefit from founder exit. Second, we find that the strong influence of venture capital backing on the rate of going public is dampened when the demographic characteristics of the TMT are controlled which suggests venture capital may be critical because of how venture capitalists alter the team. Third, functional diversity on the founding team may depress the rates of going public but it increases the rates for the TMT. This may be because we are capturing those diverse founding teams that do not receive venture capital. All these findings suggest that the relationship between VC backing and team make-up should be examined.} + There is a link between asymetric information and labor market in VC-backed startups as mentioned by \citet{davila2003venture}, \textit{Signaling theory can be extended to the information asymmetry that exists between startups and the labor market. Without a credible signal, potential employees are unable to identify high-quality startups and these startups are unable to separate themselves from their lower-quality counterparts. The existence of a round of VC can provide such a signal to the labor market (Gompers and Lerner, 1999). This signal can also impact the startup itself. A round of funding confirms the quality of the company and decreases the uncertainty about its potential success.}
  VC funding can be analysed throught the assumption of Signaling. This signal affect turnover along side of the firm.
  What is signaling theory since \citep{milosevic2020follow}, \textit{signaling theory seeks to explain how actors make such choices, by relying on signals as stimuli, which they interpret and act upon (Bergh et al. 2014). Signaling theory is used exten-sively in the entrepreneurship literature to explain how different attributes of entrepre-neurial ventures and their partners send signals of otherwise unobservable characteristics and lead external stakeholders to commit their resources to a firm (Connelly et al. 2011), or not.}
    --> From \citep{ko2018signaling} : Signaling theory is especially appropriate for new ventures in new or emerging industries, where established business models or key success factors are not known (Aldrich and Fiol, 1994). \textit{Signaling theory is concerned with reducing asymmetric information between two parties (Spence, 2002) and suggests that actors consciously and voluntarily attend to available signals to reduce perceived uncertainty (Spence, 1974). Spence (1974) originally investigated signaling effects in the employee recruitment process. He described how high-quality job candidates differentiate themselves from low-quality candidates by engaging in activities that indicate positive qualifications that are hard to imitate, such as higher education. This work prompted a volume of literature applying signaling theory to diverse disciplines as it provides insight into social selection problems under conditions of imperfect information (Connelly et al., 2011). Management and entrepreneurship scholars have found this perspective useful in that some specific signals reduce uncertainty about ventures' quality and represent future prospects in the eyes of key stakeholders, such as  top management team composition (Lester et al., 2006), board characteristics (Certo, 2003), and certifications from legitimate organizations (Baum and Oliver, 1991).} --> Even if professional investors, such as venture capitalists, possess less information asymmetry, they still make uncertain investment decisions.
    --> Why Signaling theory is linked to HC ? Since \citep{milosevic2020follow}, skills of founders or employees affect the ability to attract external funds - Certo 2003, etc. Source : \textit{Spence’s(1973) signaling model is tightly linked to human capital theory, where workers’ education acts as a signal to employers of their productive capabilities. In the entrepreneurial literature, “higher” knowledge, skills, education and experience are associated with various determinants of firm development such as opportunity recogni-tion (Gruber, MacMillan, and Thompson 2012), radical innovativeness (Marvel and Tom 2007), patenting behavior (Allen, Link, and Rosenbaum 2007) and firm legitimacy (Packalen 2007). In a large meta-analysis, Unger et al. (2011) find a robust relationship between human capital and venture success. As such, higher human capital of foundersand their teams has been found to exhibit a significant signaling effect on the ability to attract external funds (Certo 2003; Lester et al. 2006; Gimmon and Levie 2010).}
    --> The assumption of venture growth employees is truely accepted if the following design is respected : startups see in employees a way to improve value creation (so, more employees, more value). From \citet{davila2003venture}, \textit{A relevant assumption underlying this argument is that startups perceive growth in human capital as relevant to their value creation process. The power of the research design decreases if there are an increasing number of companies where the assumption does not hold. We want to thank one of the reviewers for pointing out this assumption.}
    --> Since a paper by David B. Audretsch/Erik E. Lehmann* FINANCING HIGH-TECH GROWTH: THE ROLE OF BANKS AND VENTURE CAPITALISTS --> the presence of venture capitalists enhance the growth rates of firms positively. (to be mentioned)
    --> Venture backed firms and employment growth : in \citep{davila2003venture, engel2007firm} both show that venture backed-firms have in average more annual employment growth rate. Furthermore, \citep{engel2007firm} find that in Germany, \textit{venture funded firms grow faster on average, however this difference is driven only by the technology intensive service (which includes software developers) subgroup.} Furthermore, since \citep{de2010interrelationships}, \textit{the knowledge base of employees may expand through processes of knowledge integration. Consequently, they may become more creative and innovative in daily operational tasks and problem-solving (Hansen, Nohria and Tierney 1999).} -> This means that increase over time in individual in teams increase also the mix in skills of employees : this create new conditions of performance.
    --> Since \citet{davila2003venture}, we can build on the premises theoretical framework such as agency perspective is the current model \textit{basic theoretical premise is that uncertainty and information asymmetry governs the relationship between startups and external markets—labor and financial markets. This premise is consistent with previous work that has adopted an agency perspective (Sapienza and Gupta, 1994) or risk perspective (Fiet, 1995). However, we adopt a signaling theory framework. We argue that the presence of uncertainty about the future of the startup and asymmetry of information among the players in the game enhances the value of particular events—funding rounds—as potentially valuable signaling mechanisms.}
    --> We know that VC is a double edge sword : as in \citet{sorenson2019silicon} : \textit{infusion of venture capital in a region appears associated with: (i) a decline in entrepreneurship, employment, and average incomes in other industries in the tradable sector; (ii) an increase in entrepreneurship and employment in the non-tradable sector; and (iii) an increase in income inequality in the non-tradable sector}
    --> VC has double impact on firm growth, which are positive : selection effect (or picking winners) or treatment effect. Since \citet{cavallo2019fostering}, : \textit{There is a widespread agreement among scholars over the beneficial impact of VC on portfolio companies mainly based on signaling theories (e.g. Gompers and Lerner, 2001; Hellmann and Puri, 2002). Scholars explain the positive impact of venture capitalists on new ventures based on their role in reducing problems linked to asymmetric information and moral hazard (Gompers and Lerner, 2001). VCs are investors pro-viding critical components for early stage new ventures' survival, such as contacts, information, and managerial knowledge (Bertoni et al., 2011; Sorenson and Stuart, 2001). New ventures backed by VC may benefit from a higher credibility and visibility, enhancing their chances to search for partners, attract customers and human capital (Stuart et al., 1999). These explanations fall under the “treatment effect”, where the positive association between VC involvement and new ven-tures growth is due to financial and non-financial support VCs offer to the portfolio companies (Bertoni et al., 2011). Other scholars argue that VC-backed new ventures outperform the non-VC backed new ventures, since typically VCs own superior scouting capabilities (e.g. Amit et al., 1998; Chan, 1983). This explanation is known in literature as “selection effect” or “pick winners” (Baum and Silverman, 2004; Bertoni et al., 2011; Zacharakis and Meyer, 2000). Empirical evidences show that “treatment effect” may result as prevailing over the “selection effect” (see Bertoni et al., 2011; Colombo and Grilli, 2005, 2010).}
    --> VC has an important role in firm growth for bringing governance mechanisms, processes, especially for new venture growth : \textit{Other theories supporting the positive role of VC on new venture growth stem from governance and finance literature (Dutta and Folta, 2016). VC monitor their portfolio companies through efficient contract covenants and board membership. This enhances the establishment of a structured governance and formal procedures, which is something new ventures in their early stage usually lack and need. Implementing effi-cient governance mechanisms has proved to positively impact growth and performance of companies (Sapienza, 1992). Not only the VC governance, but also the investment structure and process may support the positive role of VC on new ventures growth. VCs are known as investment funds with high incentives to exit in a time frame (Berglof, 1994) which is typically not over 10 years. As a result, they are prone to achieve exit and positive returns as such the entrepreneurs on which they have committed their funds. This may enhance and facilitate the new venture development process (Dutta and Folta, 2016). Based on the aforementioned theories, the literature showing that VC positive impacts new ventures growth is abundant (e.g. Chemmanur et al., 2011; Audretsch and Lehmann, 2004; Puri and Zarutskie, 2008). Moreover, studies revealed that the positive effect of VC to the growth of new ventures is especially evident in early stage and hi-tech new ventures (e.g.Bertoni et al., 2011; Colombo and Grilli, 2005; Stuart et al., 1999). Davila et al. (2003), while grounding their study in the signaling theories, highlight that VC investments may result beneficial founders' growth plan especially in markets with high uncertainty and complexity.}

  \item Reason #4 : depend on life-cycle phase and labor market : Since \citet{davila2003venture} \textit{demands on a team may differ at different developmental stages (cf. Birley and Stockley, 2000). Possible differences in requisite team characteristics at different devel-opmental stages have been noted in the evolutionary literature (cf. Aldrich, 1999), but such speculations have not been verified empirically in the literature on entrepreneurial teams.}
\end{itemize}

Consequence of turnover ? See \textit{grillitsch2020does} : \textit{recruiting new skills is far from a frictionless process and comes with heavy integration costs (Lockett, Wiklund, Davidsson, & Girma, 2011) due to organizational rigidities and path dependence (Beckman, 2006; Beckman & Burton, 2008), founder imprinting (Judge et al., 2015; Leung, Foo, & Chaturvedi, 2013; Marquis & Tilcsik, 2013), and the danger of threatening es-tablished routines (Beckman, 2006; Beckman & Burton, 2008; Guenther, Oertel, & Walgenbach, 2016; Hannan, Baron, Hsu, & Koçak, 2006).} --> Thus there is a cost of integration related to turnover : \textit{the costs of integration are likely to change over time, implying that there should be an opimal time for recrutiing. While existing theories are able to inform the question of timing to some degree (Beckman & Burton, 2008; Gjerløv-Juel & Guenther, 2019; Guenther et al., 2016; Hoang & Gimeno, 2010; Wiklund, Baker, & Shepherd, 2010)} And this depend on life cycle. As mentoned by \citet{grillitsch2020does}, -> adding novel skills is a negative function of firm age. --> Intrgration costs change as the venture mature over time !!

%_________________________________________####### FIGURE EXPLAINING HYPTHESIS TESTED ########

\section{Methodology}

\subsection{Samples}

Why SaaS ? --< Exploring Business Model Changes in Software-as-a-Service Firms Eetu Luoma(&) , Gabriella Laatikainen, and Oleksiy Mazhelis \textit{This paper contributes to the growing body of literature on Software-as-a-Service (SaaS). SaaS is one of the layers of cloud computing services [3, 28] and the term is used to designate standard applications delivered over the Internet [20, 37]. Choudary [10] submits that the SaaS model is associated with subscription-based revenue logic and, on that account, SaaS would entail different means of software licensing and way of charging customers compared to the traditional software business models. The Software-as-a-Service (SaaS) firms are claimed to radically change the software business setting, by breaking down the positions of big proprietary software vendors [2]. It is therefore surprising that the consideration of SaaS firms business model in the extant literature is mostly limited to their core product offering and their revenue logic. Beyond this point, the contemporary literature does not provide much more empirical evidence about how the software firms have organized their business model to develop and deliver the SaaS offerings. Specifically, the absence of empirical research on SaaS firms’ business models suggests a gap in understanding the changes in software firms’ business model to encompass the possibilities of cloud computing technologies and perils of the competitive environment.}

At first, we should explain why we focus on 2009-2019. As in \citet{taylor2006superman}, we can argue that 1/ we want substancial time (long period) and 2/ avoid a/ the crisis and b/ the covid to avoid the noise of these exogenous events. As explain in their introduction to data and sample : \textit{We analyzed data on comic books published from 1972 through 1996. We chose 1972 as the beginning year for our data, as the censorship of comics by CCA, and hence constrained innovation, ended in 1971. We ended the study in 1996 for two reasons. First, we wanted substantial time to have elapsed from the end of our data set that early valuations resulting from company marketing strength or speculation would no longer affect the perceived market value of a comic. Second, after 1996, the comic bubble generated by an influx of speculators (a consequence of widespread recogni-tion of comic art as valuable) burst, and the indus-try saw massive structural changes. Many publish-ers went out of business, and even Marvel filed for bankruptcy (Duin & Richardson, 1998). To avoid this noise in the data, we stopped the study before these structural changes occurred.}

Second, we should mention the unit of analysis : individuals or aggregated. See : Parcel, T. L., Kaufman, R. L., & Jolly, L. (1991). Going up the ladder: multiplicity sampling to create linked macro-to-micro organizational samples. Sociological methodology, 43-79. --> \textit{We argue that researchershould use representative samples to address many issues and that long-standing interest in the connections between macrolevel and microlevel processes is also central to organizational analysis. Our literature review suggests that designs that link organizations to suborganizational units or members have deficiencies involving atypicalness of cases studied or inadequate and unreliable data on organizations derived from more representative samples of individuals. Instead we advocate using a form of multiplicity sampling in which the probability of an organization's inclusion in the sample is proportional to the number of workers it employs. To illustrate this design, we use data from a study of the impact of technology and technological change on workers and their work outcomes. In our study, wefirst obtained a representative sample of workers, who provided data on their current or most recent places of employment. We then interviewed informants within those organizations to obtain organizational data. We demonstrate that our sample of establishments does not differ significantly insize or industrial affiliation from the distributions reported in archival data when the survey data are provided by the CEOs of the establishments but that workers and managers are less reliable sources of these data. We discuss the limitations of this design, as well as its potential to inform a number of issues critical to organizational analysis and macro-to-micro relationshi}

Third, we should explain how we went from 1100 startps to 606 startups : 1/ HQ location 2/ number of founders at the beginnig --> This is the screening process.

Since \citep{marvel2016human}, we focus on longitudinal \textit{the usefulness of this theory is tied to the ability of researchers to identify patterns of causality. Early reviews within the entrepreneurship literature called for hypothesis testing and regression analyses to distinguish among factors related to new firm survival (Low & MacMillan, 1988). Formal hypotheses and regression analyses have become the norm within this stream. Consequently, research could benefit from establishing causal linkages among variables through the use of longitudinal studies.} As such, longitudinal studies are preconised by scholars Since : \citep{steffens2012birds}: \textit{Further studies could also consider a \textit{FULLER} spectrum of Human capital characteristics of venture team members and employees, including detailed career histories.}

--> longitudinal studies (Evans and Jovanovic, 1989; Evans and Leighton, 1989; Black et al., 1996) studies have shown that the likelihood of being self-employed is higher for individuals with great net worth.
--> Cooper et al. (1994) in a longitudinal study of a large sample of US new ventures, show that high growth firms are more frequently created by more educated individuals.
\\

Argumenter sur le choix des VC backed firms de l'échantillon : since \citep{gruber2012minds} \textit{we selected VC-backed start-ups because these firms are strongly performance oriented and have the objectives, and potential, to grow rapidly and generate high levels of new wealth. We will comment on any limitations that the choice of our sample may entail in the Discussion section}. + context : \textit{The maturing European VC scene is, in general, closely modeled on the U.S. example (Franke, Gruber, Harhoff, & Henkel, 2008).}.

Since \citet{beckman2007early}, \textit{The focus on firms within a single region allows us to hold constant key labor market and environmental conditions. Within the region, we focused on industries engaged in computer hardware and/or software}

- Innovation context --> We are since the WOrld Economic Forum in one of the three case : factor-driven/efficiency-driven/innovation-driven economies : each stage is different in terms of complexity (Schwab, 2010)

\subsection{Context}

- Le financement en VC des startups en France serait sous optimal pour au moins 2 raisons : \\
•	les mecs de la BPI sont des administrateurs, pas des “preneurs de risque” \\
•	Coût d’agence / coordination trops forts •	\\
En d’autres termes, les profils (CS-Skills-xp) des managers VC affectent la performance des fonds VC (portfolio) et leur capacité à lever de l’argent à leur tour dans un contexte de marché sous optimal (FR) et caractérisé par une pression des élites corporatistes.\\
•	In FR, VC is doped by governement (cf . Lerner 1999 :"The government as venture capitalist: The long-run effects of the SBIR program" --> importance of localization effects) --> Context in France, since \citep{milosevic2020follow} \textit{European governments have created abundant funding programs and diverse incentives to boost the supply of VC. In France, for example, the CEO of Bpifrance1 claims that this public bank directly or indirectly finances 95/100 of VC-backed enterprises and 99/100 of seed capital-backed ventures (Lejoux 2013).} + We study the French VC market because it is one of the largest, most structured and most dynamic in Europe (Invest Europe 2018).5 At the same time, it is also one of the least successful in Europe, in which follow-on financing is particularly inefficient. Moreover, the French VC industry and especially its fundraising – be it from institutional and/or public investors – is marked by elite networks (Milosevic and Fendt 2016) and is thus likely to be a good setting to measure the impact of networks (such as banking) on raising syndicated follow-on funds.
•	VC in germany, may be similar to FRance : Since \citep{gruber2012minds} \textit{But should a VC in Germany decide to invest in a particular technology start-up, the KfW Mittelstandbank (a bank owned by the German government and the federal states and assisted by the German Federal Ministry for Economics and Labour) may be approached to become a co-investor. Through its start-up fund, the KfW participates in early-stage investments up to the amount provided by the lead investor and subject to the same economic conditions. By leveraging private equity, the KfW thus incen-tivizes private-sector investment in technology start-ups. Hence, a technology start-up in the United States that would not receive funding because the VC is unwilling or unable to pro-vide the required amount of funding might receive funding in Germany because the VC can obtain matching funds via co-investment by KfW.}. --> Sould investigate the role of the KfW

--> Scraped methodology ; Linkedin, one of the most comprehensive professional social network in the world,

\subsection{Variables}

--> What is attempted in this project ?\\

The objective of the study is to investigate the link between full team employees Human capital composition and performance of software companies (regarding indicators such as the advent of funding events or the viability of a firm). High tech companies are not always software companies. Hightech can be electronics, biotechnoogy, etc. Thus we focus on Software companies. As such, high growth potentiel software startups have specific predictors of success, differents from software established firms or new ventures. Since \citep{hathaway2013tech}, employees in high tech sector are defined as \textit{the group of industries with very high shares of employees in the STEM fields of science, technology, engineering, and math.}. \\

As such, we chose several variables to understand firm performance. As this topic of interest has been widely adressed, we review a text summarizing all the indicators. Since \citep{marvel2016human}, \textit{The most common human capital construct investigated was work experience, representing 39.9/100 of the constructs. This was followed by education, representing 26.6/100 of the human capital constructs. Consistent with past findings, investments in education and experience were clearly the most pervasive types of constructs employed (Reuber & Fischer, 1994). Examples of typical education measures include years of education or completion of a university or technical degree. Examples of experience-based measures include past work in an industry or the number of previous management positions held. The third most common measure was entrepreneurial experience (19.8/100), such as past start-up experience or prior business ownership, which reflects a specific type of human capital from the entrepreneurial context. Less commonly assessed measures included demographics such as age, whether family members were entrepreneurs, or gender. A handful of studies included cognitive and/or psychological measures as key aspects of human capital. For example, locus of control and achievement orientation were both included as human capital constructs, which suggests the blurring of boundaries among human capital and psychological constructs.}\\

Since Audretsch et al., 2015, in line with the city-type findings, we found a negative relationship between low human capital and urban economic development, reinforcing the importance of human capital from previous research (Lucas 1988; Glaeser et al. 1995). HOWEVER, Human capital, proxied by formal educational attainment, has no effect on economic development regardless of market size (Rodriguez-Pose and Crescenzi 2008). Thus, we should go fine-grained.

In each of the variable described under, we must take into account team-diversity of experiences and education. As such, we can find the argumentation in \citet{taylor2006superman} : \textit{Diversity of knowledge in teams has been termed “deep-level diversity” (or “cognitive diversity”) to distinguish it from diversity in surface characteris-tics such as the demographic variables of age, gen-der, and race (Harrison, Price, Gavin, & Florey, 2002; Jehn, Northcraft, & Neale, 1999). Although the theory of knowledge combination holds that diverse knowledge components generate perfor-mance variance (Fleming, 1999), the information processing perspective on team diversity holds that greater cognitive diversity leads to higher perfor-mance potential.}

Mathod research design we should use ?
--> https://en.wikipedia.org/wiki/Event_study Event study : An event study is a statistical method to assess the impact of an event on the value of a firm.

-------> Inputs/ INDEPENDENT VARIABLES\\

In the literature, various conceptions of human capital are considered. First, \textit{"general human capital"} that does not directly relate to venture tasks (e.g., formal education [technical vs. scientific vs. economic vs. managerial] and employment experience [technical vs. commercial]). Second, \textit{"specific human capital"}, which refers directly to market and industry specific context. In these two first distinctions, education or work experiences help individuals to discovering and creating entrepreneurial opportunity \citep{marvel2016human}, get a better understanding of a situation, to manage better their time, to access networks, increase problem-solving behavior, employ and create strategies. As a consequence, it is considered that founders, TMTs or employees represent a stock of unique and idiosyncrasic combination of resources \citet{becker1964human}. Third, \textit{"task-related HC"}, which relate to the current task of the venture e.g., start-up experience, industry experience, business skills \citep{gibbons2004task}. Also, another conception made by entrepreneurship scholars distinguish HC investment (i.e., education and experience) from outcomes of HC investment (i.e., the skills, knowledge, and abilities that derive from education and experience) \citep{unger2011human, marvel2016human}.

As a consequence, on their side, organizational demographers frequently raise a debate that refers to homogeneity vs. diversity ant its impact on firm performance. To date, this debate led to mixed conlusions \citet{beckman2007early}. On one side, some suggest that team members homogeineity in HC improve communication and reduce team conflict (Ancona and Caldwell, 1992 ; Watson 1993). Indeed, \textit{communication in large groups has a process cost that reduces group out-puts (Kurtzberg & Amabile, 2001; Steiner, 1972).} \citet{taylor2006superman}. On the other side, others say that diverse team are more effective because of their skills complementary (Meakin and Snaith, 1997).

As in \citet{ghassemiautomated}, \textit{•	Prior work on the qualities of teams that predict success have studied the effects of demographics (age, aca-demic status, academic specialization, and prior work expe-riences) (Visintin and Pittino 2014; Delmar and Shane 2006; Beckman, Burton, and O’Reilly 2007). + (Eisenhardt and Schoonhoven 1990). •	In the context of entrepreneurship competitions specifically, the relationship between team demographics and competition outcomes has also been studied (Der Foo, Wong, and Ong 2005).}

- Variable (1) : education
There must be a distinction between general (PhD, master, bac+2, etc.) and specific educ HC (economic, sciologie, etc.)
\begin{itemize}
  \item Description : Education is one of the most studied explanatory variable. Education deal with problem solving ability, skills. since \citet{colombo2005founders}, \textit{ some studies suggest that there is a significant positive relationship between human capital variables and survival (e.g., Bruederl et al., 1992; Evans and Leighton, 1989; Gimeno et al., 1997), others have reported insignificant relationships (Bates, 1990; Cooper et al., 1994; Kalleberg and Leicht, 1991; Stuart and Abetti, 1990).}
  \item Precautions : however, be carefull with education : if an entrepreneur is so much educated, there is an opportunity cost --> staying in a big company is easier for him because he may get paid more at short term. --> founders' education level (Becker, 1964; Shane and Stuart, 2002)
  \item Empirical results : since \citep{gruber2012minds}, \textit{research on top management teams suggesting that educational diversity is positively related to corporate diversification (e.g., Wiersema & Bantel, 1992) and innovation (e.g., Bantel & Jackson, 1989).}.
  \item Empirical results : Since Paul Westhead and Marc Cowling -"Employment Change in Independent Owner-Managed High-Technology Firms in Great Britain", entrepreneurs background (individual capital humann) as education or finance at the begenning enhance firm hability to create employment.
  \item Empirical results : Since \citep{engel2007firm}, firm size, high education degrees, and patents at foundation date, has a positive influence on the probability of being venture funded or not. --> \textit{"firms in R&D oriented industries are venture funded with higher probability."} However, this conclusion probably reflect recent dynamic evolution of the German venture capital market.
  \item Empirical results : From \citep{ko2018signaling} \textit{:"the founder's level of education not only directly influences venture performance but also provides a positive signal to investors. One of reasons is that founders with higher education generally will not allow themselves to be connected to or involved in lower-quality ventures, particularly due to the associated opportunity costs (Gimeno et al., 1997)."} --> HC is a signal for investors as they want to invest in people that will give back their interest. as well as industry knowledge as founder prior industry and founder previous xp in startup CEO is a goood indicator for product commercialisation hability : reduce uncertainty and environment product uncertainty
  \item New avenue of research : from \citep{marvel2016human} : \textit{the length or completion of formal education is appropriate, but encourage future studies to consider other approaches, such as the discipline (i.e., type) or diversity of human capital invest-ments (i.e., engineering, liberal arts, natural science, or social science degrees).}
  \item New avenue of research : from \citep{marvel2016human} : \textit{Similarly, previous research has taken an approach of task versus non-task human capital and considered formal education and employment as non-task because they do not directly relate to venture activities. However, this simpli-fication ignores the possibility that the formal education may be a bachelor degree of entrepreneurship or the work experience is within a start-up organization. These examples illustrate a gray area in terms of prior start-up experience, education, and employment experience that dichotomous approaches fail to capture. Future research should employ finer grained measures that reflect more precise degrees of variance.}
  \item Since \citet{colombo2005founders}, founders’ years of university education in economic and managerial fields and to a lesser extent in scientific and technical fields positively affect growth while education in other fields does not.
  \item Since \citet{colombo2005founders}, \textit{While this variable (education) is generally found to be positively related to the likeli-hood ofsurvival ofnewfirms (see Bates, 1990; Br¨uderl et al., 1992; Gimeno et al., 1997), results concerning growth are less robust.}
  \item in a study very similar to our, \citet{grillitsch2020does} use ONLY education background as adding a new skill : they don't focus on Outcomes => Marvel and cie. recommandatoons are not followed. However, they argument in this sense (cognitive framework and educational background) \textit{Schubert and Tavassoli (2019) argue that the reason for the usefulness of educational backgrounds is that they create cognitive frameworks which not only determine what an individual currently knows or is able to perform, but rather, educational backgrounds affect and filter the information an individ-ual perceives as useful or valid. Therefore, cognitive frameworks resulting from education develop cumulatively and are self-enforcing. As an example of such cumulative development of cognitive frameworks resulting from education, several authors have highlighted remarkable differences in competences, solution approaches, and skills between engineers and scientists, which do not vanish as individuals age (Allen, 1984; Allen & Katz, 1992; Faems & Subramanian, 2013)}
\end{itemize}

- Variable (2) : Prior industry / professional experience
There must be a distinction of general (years before funding a firm, entrepreneur age) and specific (business xp in the same sector /industry of the new firm as a proxy of industry-specific HC xp) mentioned as industry-specific experience (Tyebjee and Bruno, 1984) in the litterature.
\begin{itemize}
  \item Empirical results : since Gruber et al. 2012, they found that both technological experience and marketing experience have a negative effect. This provides rare quantitative evidence indicating that individuals in technology tend to show less appreciation for market-related issues (Dougherty, 1992).
  \item Empirical results : since \citep{cooper1994initial}, Last xp is so important \textit{The organization where the entrepreneur was located just prior to starting a firm might be viewed as an “incubator” where a num-ber of valuable experiences take place (Cooper [1985]).}
  \item Empirical results : regarding variable (1) and (2), as mentioned in \citep{steffens2012birds} \textit{research suggest the impact of HC on team performance is mixed : work experience and education diversity has positive impact on performance BUT functional homogeneity improve team perf as well as homogeneity of industry... So in this paper, they argue that it depend on the lifespan of the startup (early Vs. later)}
  \item Avenue of research : from \citep{marvel2016human} : \textit{work experience is commonly assessed by the years of industry experience. To more fully explore investments in experience, the types and diversity of work experience should be considered. For example, experience in R&D, marketing and sales, or previous leadership roles may have varying implications. In many cases, studies have applied dichotomous approaches to human capital such as prior business ownership or task versus non-task. We believe these operationalizations oversimplify human capital and limit our understanding.}
  \item Empirical results : from \citet{beckman2007early}, \textit{Entrepreneur-ship scholars have demonstrated that the quality of the team’s past experience benefits their firm (i.e., Burton et al., 2002; Chandler and Hanks, 1998; Schefczyk and Gerpott, 2000).} As in this study, We rely on the job title to determine whether a person is a member of a team. We do not know whether the person held equity in the venture. However, we go well furter Beckman since they have 1744 executives, 149 firms, and 2,69 position / individual. We have a lot more.
  \item Since \citet{colombo2005founders}, Similarly prior work experience in the same industry ofthe new firm is positively associated with growth while prior work experience in other industries is not.Furthermore, it is the technical work experience of founders as opposed to their commercial work experience that determines growth.
  \item since Since \citet{colombo2005founders}, general prior work experience has no robust results (regarding its impact on growth or survival, it depends)
  \item since Munoz et. al 2015 previous xp of founders have a positive impact on startup perf if this xp is related to industry. \textit{team resource heterogeneity has a positive impact on profitable firm creation. Moreover, this positive effect is greater as the team has more experience in the industry in which the new business will compete.}
  \item since \citet{taylor2006superman}, \textit{The role of career experience in generating unique individual stocks of knowledge is especially important for work done in creative or problem-solving teams in settings such as consulting (Hansen, 1999), product development (Hargadon & Sutton, 1997), and creative industries (Miller & Shamsie, 2001). In such work, individuals develop their own knowledge and network ties with other knowledgeable persons, both of which help them to retrieve and apply knowledge components useful for a given task. As a result, the size of a team drives diversity in knowledge and ability to innovate (Jackson, 1996).}
  \item since \citet{colombo2005founders}, about founder and their xp in industry specific XP \textit{Findings relating to founders’ industry-specific human capital are more robust. Br¨uderl et al. (1992) document that the failure rate of Bavarian new firmsis significantly lower if founders have business expe-rience in the same sector of the new firm. In addition, in a univariate analysis this variable is shown to have a significantly greater value for fast growing firms than for other firms; nevertheless, such relation is not replicated in a multivariate logit analysis (see Br¨uderl and Preisend¨orfer, 2000).} ALSO, in the next paragrpah, \textit{Cooper et al. (1994) find that industry-specific know-how contributed to both survival and growth of their sample firms. Siegel et al. (1993) show that in a sample of around 1600 Pennsyl-vania start-ups, the fact that the entrepreneurial team had prior experience in the same industry of the new firm was the only discriminating factor between high-growth and low-growth firms.}
  \\

\end{itemize}

- Variable (3) : previous funding experience
\begin{itemize}
  \item Description : prior founding experience as for example (CEO/ co-founder) (Hsu, 2007). Employees seldom (rarement) take alternatives as such founding experience matters.
  \item Empirical results :
  \item Avenue of research : from \citep{marvel2016human} : \textit{As an example, the degree to which previ-ous business ownership experience applies to an entrepreneurs’ future situation will likely be impacted by how similar the opportunity, industry, market, or product was to the previous venture experience.}
  \item Since \citet{colombo2005founders}, The fact that within the founding team there are individuals with prior entrepreneurial experiences also results in superior growth.
  \item since \citet{taylor2006superman}, knowledge evolve alongside with experience : \textit{Through their career histories, individuals can be-come proficient in multiple knowledge domains and motivated to combine them.} As such, previous funding experience should be a vector for such performance : \textit{teams with individuals who each hold diverse knowledge domains will be likely to combine them. In addition, teams allow individu-als who do not hold diverse knowledge to become exposed to it to through interaction with other members (Nonaka & Takeuchi, 1995).} --> \textit{teams whose members have and share di-verse knowledge can obtain higher levels of indi-vidual and team creativity (De Dreu & West, 2001).}
  \item Since \citet{colombo2005founders}, Entrepreneurial HC is a good indicator. \textit{how to manage a new firm, that is entrepreneur-specific human capital; this is developed by founders through “leadership experience” (Br¨uderl et al., 1992) obtained either through a managerial position in another firm or in prior self-employment episodes. In this paper, we conform to this distinction.}
  \item since \citet{colombo2005founders}, results of having a previous entreprenerial xp and its effect on growth is mixed results. \textit{Bates (1990) and Br¨uderl et al. (1992), respectively, find no evidence that individuals’ prior managerial and self-employment experiences have any impact on new firms’ failure rates.} + \textit{Gimeno et al. (1997) show that while prior managerial and entrepreneurial experiences positively influence new firms’ economic performances, they have no significant impact on survival.} + Chandler and Jansen (1992) highlight that the number of businesses previously initiated by founders and the years they previously spent as owner–manager do not affect new firms’ growth; on the contrary, the years ofgeneral managerial experience in another firm are positively correlated with self-perceived manage-ment competence, which in turn is a predictor offirms’ growth.
  \item Since \citep{marvel2016human}, Unger et al. (2011) suggest the entrepreneurship–success relationship is higher for outcomes of human capital than for investments alone because investments are indirect indicators and thus one step removed.
  \item As such, for example in \citet{reese2020should}, they focus on outcome of HC for founders (xp as an entrepreneur) --> \textit{Founders' human capital can be conceptualized on the basis of their investments in human capital (i.e., education and experience) or the outcomes of these investments (i.e., the skills, knowledge, and abilities that derive from education and experience) (Becker, 1964; Kor, 2003; Unger et al., 2011). While the latter is often neglected in the extant entrepreneurship research because of the difficulties of empirical investigation (Marvel et al., 2016), we focus on these outcomes of human capital investments—more specifically, the founding team's com-petencies in the form of their specific skills, abilities, and knowledge. These outcomes are what the founders are able to deliver to the entrepreneurial process, while investments in human capital only allow research to derive indirect derivations of the skills, abilities, and knowledge that might have been established in the course of acquiring educa-tion and/or experience (Marvel et al., 2016).}
\end{itemize}

- Variables (4) : Demographical (age)
\begin{itemize}
  \item Description : age of an individual - can be related to degree obtention
  \item Empirical results : To this debate, from \citep{steffens2012birds}, 2 studies show that age of TMT has ambiguous outcomes : it does not increase innovation, but performance. As such, TMTs age diversity is positively related to turnover (Tsui and O'Reilly 1989, Wiersema and Bird, 1995).
\end{itemize}

- Variable (5) : firm size
  \item size of the company is a dépendent and indep variable. If we use Size as an indep : see \citet{hanks1994tightening}  to justify why we should use natural logarithm of size \textit{Organization size (Sizelog) was measured by the natural log of the organization's reported total employ-ment at the end of the 1987 fiscal year. The natural log of this measure minimizes the effect of skewness in these distributions (Blau & Schoenherr, 1971; Khandwalla, 1977; Hinkson, Hinings, & Turner, 1968).}

- Variable (6) : background affiliation
\begin{itemize}
  \item Beckman refers to 2 kind of capital : bounding (overlap of xp) and bridging (diversity of xp) capital --> Les deux sont complémentaires.
  \item Developed by \citep{beckman2007early}) \textit{background affiliation — a new kind of team demographic characteristic — that may be particularly relevant for the success of young firms. Managers bring a good deal of tacit knowledge with them from their prior firms about how to organize and manage work processes, and this knowledge is likely to differ even between two firms in a similar industry. We look at two aspects of a team’s background affiliations based on prior company affiliations (that is, the companies in which the team member’s worked before joining the firm). We examine affiliation diversity (i.e., how many unique companies the team members have worked for) and affiliation overlap (i.e., proportion of prior past company experiences at the same company).}.
  \item Furthermore, regarding bridging capital : background affiliation \textit{In entrepreneurial settings, as in other settings, contacts with others in the industry and financial community are positively associated with a firm’s valuation at IPO and access to VC (Burton et al., 2002; Higgins and Gulati, 2003; Stuart et al., 1999). The more affiliation diversity of the team, the greater its range of experience and contacts, and this should help firms be more successful in both attracting VC and achieving IPO. Affiliation diversity should both increase the direct and indirect contacts that can be utilized as well as increase the unique insights and knowledge of the firm about how to reach these milestones (Burt, 1992; Smith et al., 2005).}
  \item Furthermore, regarding bounding capital : \textit{teams who have affiliation overlap (i.e., having worked for the same firm) may be able to communicate effectively with each other and have a common frame of reference, especially since these individuals have chosen to work together in the new venture. Founding team members who worked together before appear to be more effective and have greater trust (Eisenhardt and Schoonhoven, 1990; Roure and Maidique, 1986), and trust is an important component of social capital (Coleman, 1988).}
\end{itemize}

- Variable (7) : previous common experienced
\begin{itemize}
  \item the fact that people have worked in the same company before (or in projects, for example), induce more performance as there is less propency to conflict as they know each other. See : \citet{taylor2006superman} : \textit{teams with many prior collaborative projects have better communication and are thereby more likely to fulfill the goal of obtaining more creative outputs. These teams are also more likely to develop standardized practices for operation, which result in higher mean perfor-mance outcomes (Gilson et al., 2005).}
\end{itemize}

-------> Measuring growth (outputs) / DEPENDENT VARIABLES\\

Since \citet{marvel2016human}, \textit{traditional venture outcomes that include survival, innovation, and growth in sales, profitability, or employment among others.}\newline
Innovation measure : référence = Ruef 2002. (island, weak ties, etc.)
See in Chandler and Hanks 1993, Measuring the performance of emerging businesses: A validation study --> there is a lot of variables, but 2 are fine to study : Growth and sales (business volume)

- Variables (1) : Team size (total employment)
\begin{itemize}
  \item Description : the size can be at the beginning of the venture or the evolution. Depending on the study, in \citep{davila2003venture}, the size of the team is considered has a proxy of growth. From \citep{unger2011human} \textit{firm size as a performance indicator for entrepreneurial firms which start from zero (Eisenhardt and Schoonhoven, 1990).}
  \item Empirical results : Since \citep{steffens2012birds}, \textit{team-founded ventures tend to be over-represented among high performers (Bird 1989; Kamm et al. 1990; Cooper and Gimeno-Gascon 1992; Timmons 1990)} because they get access to complementary resources, etc. A consequence to that is a better performing ?
  \item as mentioned above : \citet{taylor2006superman} : \textit{teams with many prior collaborative projects have better communication and are thereby more likely to fulfill the goal of obtaining more creative outputs. These teams are also more likely to develop standardized practices for operation, which result in higher mean perfor-mance outcomes (Gilson et al., 2005).}
  \item size of the company is a dépendent and indep variable. If we use Size as an dep : see \citet{hanks1994tightening}  to justify why we should use natural logarithm of size \textit{Organization size (Sizelog) was measured by the natural log of the organization's reported total employ-ment at the end of the 1987 fiscal year. The natural log of this measure minimizes the effect of skewness in these distributions (Blau & Schoenherr, 1971; Khandwalla, 1977; Hinkson, Hinings, & Turner, 1968).}
\end{itemize}

- Variable (2) : Growth rate (employment growth)
\begin{itemize}
  \item Description : Employment growth is a valuable dependent variable. From \citep{visintin2014founding} \textit{We measured USOs performance through growth indicators, since growth is considered as the most appropriate dimension of performance in new ventures (Weinzimmer et al., 1999). Following methodological recommendations and recent work on USOs performance (Delmar, 1997; Wennberg et al., 2011)we adopted a multidimensional approach and assessed growth over two attributes: sales growth and employment growth over the three years after the start up year. It has been argued that for high-technology start-ups, employment could increase before any sales occur. Moreover, some authors contend that employment is a much more direct indicator of growth since it proxies the increase in managerial complexity (Delmar et al., 2003)Both indicators were expressed in absolute terms: sales growth as the natural logarithm of the difference between sales in of the start-up year (t0) and in the third year after the start up year (t3) and employment growth as the difference between the number of employees in the year of start up and three years after. Since we employed the natural logarithm of sales, we added the value one to the amount of sales to correct for initial values equal to zero. We chose a three year time span since it is the one that is most frequently adopted in studies on early growth performance ofnew ventures (e.g. Dobbs and Hamilton, 2007; Hansen and Hamilton, 2011). We controlled for the influence of size on subsequent growth performance by including employment and sales at start up as control variables (Delmar, 1997).}
  \item Growth rate measure of employees : \citet{hanks1994tightening}, they use growth rate to measure the stage-life cycle. We can use it for employees also, as : \textit{The growth rate measure (Employee Growthy) reflects organization growth for the firm's most recent year of performance. It was calculated, utilizing self-reported em-ployment data, based on the following formula: (Total Employees 1987-Total Employees 1986) Total Employees 1987 Employee Growths = This measure of growth, though somewhat unorthodox,}
  \item wHY use this variable instead of sales measure ? Since \citet{hanks1994tightening}, \textit{Employment-based measures of organization size and growth, as opposed to sales-based measures, were utilized in formation of the clusters for two reasons: first, it was believed that organization structural response would be more closely related to number of employees than sales; and second, the disclosure rate of respondents was higher for employment figures than sales figures, thus allowing us to retain more firms in the analysis.}
  \item since \citet{grillitsch2020does}, Employment growth has pitfalls : \textit{Usually, employment growth is regarded as inferior. First, firms can meet increasing demand by hir-ing staff but also through other means such as subcontracting (Delmar, 2006). Second, employment growth neglects the fact that firms can have varying capital intensities. Sales data provides therefore a more direct measure of the growth of start-ups than employment.}
\end{itemize}

-------> Measuring successfulness (outputs) / DEPENDENT VARIABLES\\

- Variable (1) : funding events (VC)
\begin{itemize}
  \item Description : as in \citep{davila2003venture}, we could consider the event-signal here as funding events - in this case, as a growth proxy indicator. Signaling theory is useful for describing behavior when two parties (individuals or organizations) have access to different information : in this case, a firm and an investors. This is higly related to agency theory. Venture capital = financial capital. Financial resources help to create a buffer in wavy environment and help financing new strategies but also to create competitive advantages and difficult-to-imitate innovations. Every funding date is a special moment since new objectives are set up with actionnaires --> they raise money on a business plan and growth projections. Thus, needs in human capital in order to reach performance evolve during startup lifespan.
  \item Empirical results : VC and firm performance. Since \citep{engel2007firm}, there is a dense litterature that examine empirically the relationship between receiving VC and firm performance - see (Schefczyk, 2000) for a review. For example, VC funding influence firm innovativeness (patent) : Kortun and Lerner, 1998/200 / --> service provided by VCist is useful (cf. Massimo et Grilli - scout and scan). However, the conclusion since \citep{engel2007firm} is not very conclusive.
  \item From \citep{ko2018signaling} \textit{empirical evidence show that venture capitalists consider founders' advanced educational degrees (Hsu, 2007; Roure and Maidique, 1986) when they evaluate new ventures.} // endorsements from prominent investors signal that a new venture would benefit from investors' resources, such as investors' management skills and tacit knowledge (Colombo and Grilli, 2010; Davila et al., 2003).
  \item Critics of using VC financing as a performance indocators : \citep{delmar2006does},raising money, are aspects of the organizing process itself (Delmar and Shane, 2003).
  \item In which venture do VCs invest ? since \citep{siepel2017non}, \textit{Sommer et al. (2016), who find that perceived innovativeness of a firm makes the firm more attractive to potential employ-ees.}
  \item Since \citet{beckman2007early}, \textit{VC are important markers of success because ties to venture capital investors increase a firm’s chances of survival (Shane and Stuart, 2002).} Furthermore, - to adapt to our paper, \textit{In addition, in our context — high technology firms in Silicon Valley during the 1990s — these were both salient and desirable outcomes for entrepreneurs to achieve as quickly as possible. In addition to being established in the literature and appropriate for our context, both VC financing and IPO attainment have the additional advantage of allowing us to examine outcomes across multiple high technology industries (e.g., semiconductors and....} + \textit{Obtaining VC funding and going public together represent the most significant milestones in the life of a young start-up firm (Shane and Stuart, 2002).}
\end{itemize}

- Variable (2) : firm survival
\begin{itemize}
  \item Description : Since \citep{marvel2016human} \textit{firm survival is a dependent variable construct that has been commonly used in entrepreneurship research (e.g., Cooper, Gimeno-Gascon, &Woo, 1994; Dencker, Gruber, & Shah, 2009). This approach may be particularly useful in instances where traditional performance measures such as degree of profitability, sales, and return on assets are inappropriate./ many entrepreneurial objectives are more easily assessed as categorical rather than continuous (e.g., survival, start-up, new product creation, entry/exit, etc.).}
  \item Careful : pay attention between dissolution / failure : from \citep{unger2011human} \textit{not to include studies reporting firm dissolution as the dependent variable. Such measures are often ambiguous because they may or may not signify business failure (Headd, 2003).}
  \item see a paper called : New Venture Survival: A Review and Extension Aracely Soto-Simeone , Charlotta Siren´ 1 and Torben Antretter2
  \item as mentioned by \citet{grillitsch2020does}, survivor is a complementary variable joint to firm growth that can be measure. \textit{Another aspect concerns the choice of the independent variable. We have chosen firm growth as our core variable of interest. While early growth and scale-up are important for the success of start-ups in the long run (Cefis & Marsili, 2005; Pe'er et al., 2016), survival may be an equally (or more) im-portant performance dimension for many start-ups. Indeed, several of the papers that have analyzed questions closely related to ours (Gjerløv-Juel & Guenther, 2019; Guenther et al., 2016) have focused on survival as the final outcome. The focus on survival may also be driven by the notion of liabilities of newness, which are typically expressed in terms of high failure rates (Baum & Amburgey, 2002; Baum & Oliver, 1996; Sleuwaegen & Onkelinx, 2014). While, in our view, the notion of liabilities of newness could also be expressed as disadvantages in terms of firm growth, a natural extension of our work would be to analyze how the timing of recruiting novel skills affects firm survival. Moreover, such research would contribute to deepening our understanding of how growth and survival are linked to each other (Pe'er et al., 2016).}
\end{itemize}

-------> Control Variables / CONTROLS VARIABLES\\

2 kind of controls are used in the litterature. Since \citet{beckman2007early}, they use 3 controls at the team level (individuals), and 4 at the firm level (aggregated). As such, we reproduce here the dichotomie and add more controls used in the literature.

----------------------------------------------------------------------
Team controls : (individuals)
----------------------------------------------------------------------

- Control (1) : team size
\begin{itemize}
  \item As in \citep{de2010interrelationships} and many other studies, we can control with time size \textit{company size (number of employees) to control for possible discrepancies between the very small and small firms concerning access to financial means}
\end{itemize}

- Control (2) : prior experienced (startup or management prior xp)
\begin{itemize}
  \item As in \citet{beckman2007early}, they watched if in the founding team had a previous startup funding experience. AND of the team member had prior senior management experience
\end{itemize}

- Control (3) : team tenure (titulaire, mandat, occupation)
\begin{itemize}
  \item As in \citet{beckman2007early}, meand and standard deviation of team tenure - typical measures of tenure diversity in demographic research.
\end{itemize}

----------------------------------------------------------------------
Firm controls : (aggregated)
----------------------------------------------------------------------

- Control (1) : firm size and composition (diversity effect)
\begin{itemize}
  \item As in \citet{beckman2007early}, Firm size is measured as the number of employees at the end of a given year and is updated yearly.
  \item as in \citet{taylor2006superman}, we can control for pourcentage of diversity that affect, since the litterature, firm perfoance. as such : \textit{An important characteristic of team composition is the experience that the members have in working together. Teams go through a process of socialization that makes communication easier as members adapt to each other (Katz, 1982). The result is that low or diverse member tenures in teams create communication difficulties (Pfeffer, 1983), which increase the negative effects of member diversity on communication (Pelled, Eisenhardt, & Xin, 1999). Conversely, teams with sufficient experience to have established efficient communication can more easily utilize member diversity (Harrison, Price, & Bell, 1998; Harrison et al., 2002).}
  \item As in \citet{grillitsch2020does} we must control for size and take into account the dilemma between growth and size. \textit{we control for firm size (logarithm of sales), which accounts for the long debate on the relationship between firm size and firm growth. While Gibrat (1931) posited that growth is uncorrelated with size, a number of studies actually show that growth is affected by size and a number other factors (Grillitsch, Schubert, & Srholec, 2019). Furthermore, firm size may serve as a control for potential non-constant returns to recruitment. One argument that would postulate decreas-ing marginal returns suggests that firms recruit new employees with descending order of urgency. Thus, larger firms generally profit less from hiring. To the degree that firm age and size are correlat-ed, we may attribute effects of late hires with effects that are essentially due to hires in larger firms.}
\end{itemize}

- Control (2) : sector/industry
\begin{itemize}
  \item As in \citep{de2010interrelationships} and many other studies, we can control with sector \textit{due to potential industry differences in innovative and HR related activities, we control for sector. Dummy codes representing seven broad industry categories are created: manufacturing, construction, wholesale & retail, catering, transportation & communication, financial services and health & personal services. Industry is used as our point of reference.}
  \item As in \citet{beckman2007early},\textit{control for industry to capture differences in the ability of firms to obtain financing and go public (some industries are more likely to get funding and go public than others). We examined the effect of all the industry dummy variables}.
  \item as in \citet{grillitsch2020does} we can control for this : industry year-fixed effect.
\end{itemize}

- Control (3) : funding (be Seed, VC, etc.)
\begin{itemize}
  \item As in \citet{beckman2007early}, they control for VC funding for IPO (the ones that raised VC funding are more likely to get IPO). In our case, we can control for 1/ seed (the one that get VC funding are those that get seed). We can also control for the fact that founders previously raised money in a previous company ! (we lend to rich people or trustful people)
\end{itemize}

- Control (4) : Amount of team data (against biais of completedness)
\begin{itemize}
  \item As in \citet{beckman2007early}, \textit{Because we collect our career history data via archives, there is a potential for an inadvertent bias in our data where we have systematically less information about the career histories of executives who are from firms that receive less media coverage. In order to control for these firm-level effects on the quality of our career history data, we include a control variable that represents the average number of prior positions per person collected for each firm.}
\end{itemize}

- Control (5) : turnover
\begin{itemize}
  \item Description : Since \citep{beckman2007early}, \textit{decomposition of turnover allows us to understand the mechanisms by which turnover and tenure heterogeneity matter.} + \textit{In examining the data in more detail, it appears that firms with venture capital backing are those that benefit from founder exit.}
  \item as in \citet{grillitsch2020does} we can control for turnover : \textit{A key control variable is the share of employees leaving or entering the firm each year: share recruitsi,t = new employeesi,t/team sizei,t (3) (4) share leaversi,t = exited employeesi,t/team sizei,t new employeesi,t is the number of new employees of firm i in year t and exited employeesi,t is the number of employees who were part of the workforce of firm i in year t-1 but not in year t. team sizei,t is the total number of employees of firm i in year t. If we did not control for new or exiting employ-ees, the effects of adding new skills could be confounded with the effects of other changes in the workforce.} NB : The relation between adding a new skill and firm age IS NOT LINEAR !!!!!!
\end{itemize}

- Control (6) : general level of skills of the work forces
\begin{itemize}
  \item as in \citet{grillitsch2020does} we can control for this : \textit{(share of employees with tertiary education in total employment).}
\end{itemize}

- Control (7) : firm age : moderating factor.
\begin{itemize}
  \item as in \citet{grillitsch2020does} we can control for this : \textit{(In order to allow for an estimation of the age effect which is more flexible than the one in Eq. (5), we created a vector of 15 age dummies for firm age=1, …, 15 (AGEi,t). The dummies are interacted with 15 interaction terms of new educational the share backgrounds, creating a vector of (sh_newedui,t·AGEi,t) leading to the model in Eq (6). All other specifications are identical to the model represented in Eq (5).} --> Equation very easy to understand.
\end{itemize}

- Control (8) : number of individuals in the funding team.

----------------------------------------------------------------------
Industry level controls of VC : (aggregated)
----------------------------------------------------------------------
- Control (1) : VC supply
\begin{itemize}
  \item As in \citet{beckman2007early}, we can control for the global amount of VC funds delivered in the industry (some years, there is more VC funds in 2019 lets say, and less in 2013).
\end{itemize}

----------------------------------------------------------------------
Macro controls: (aggregated)
----------------------------------------------------------------------
- Control (1) : macroeconomic growth
\begin{itemize}
  \item As in Audresht and Acs in the paper called : New-firm startups, technology, and macroeconomic fluctuations
\end{itemize}

- Control (2) : level of unemployment
\begin{itemize}
  \item As in Audresht and Acs in the paper called : New-firm startups, technology, and macroeconomic fluctuations
\end{itemize}

\subsection{Robustness tests}

Check for various pitfall to be done : for the study to be robust !\\

- (1) Collinearity between initial conditions and future firm behavior / Multicollinearity, checked with VIF - variance inflaction factor(8.3 ?). Since \citep{de2010interrelationships}, \textit{Following Subramaniam and Youndt (2005), we centered (X ¼ 0) the employees’ human capital and HR practices variables to minimise the effects of any multicollinearity among the variables comprising our interaction effect. To be sure of the absence of multicollinearity problems after this transformation, we tested the path between innovation (dependent variable) and the transformed employee variables as well as their interaction effect (independent variables) in a simple regression. Two measures guiding us to detect multicollinearity are the variance inflation factor (VIF) and tolerance. The rule of thumb is that VIF . 4 and tolerance , 0.20 when multicollinearity is a problem (Menard 1995).} + Another computation : from \citep{visintin2014founding} \textit{consistently lower than 10 and this, as suggested by Neter et al. (1996), indicates the absence of significant collinearity problems.}\newline
As in \citet{reese2020should}, VIF + In addition, following Echambadi and Hess (2007), we conducted ana-lyses using subsamples to check for bouncing betas that would indicate harmful multicollinearity.

- (2) replicability of the model (cf. Delmar 2006)\newline

- (3) statistical significance of the model to be check (MNL-x2 \citep{cooper1994initial})\newline

- (4) assess model predictive power (Cooper 94)\newline

- (5) having various performance indicators (cf. Delmar 2006). As in \citet{grillitsch2020does}, they use 2 alternatives measures of growth : turnover vs. employment growth)\newline

- (6) survivorship bias mentionned as crucial by \citep{unger2011human} \textit{While some studies suggest that there is a significant positive relationship between human capital variables and survival (e.g., Bruederl et al., 1992; Evans and Leighton, 1989; Gimeno et al., 1997), others have reported insignificant relationships (Bates, 1990; Cooper et al., 1994; Kalleberg and Leicht, 1991; Stuart and Abetti, 1990). However, many of these studies did not distinguish between success and survival and between failure and successful closure (Headd, 2003). Our results, thus, cannot be generalized to survival and failure of business ventures.}. Also mentioned by \citet{grillitsch2020does}, survival is a major pitfall to avoid. \textit{selective survival bias explanation : since firms with consistently negative growth face a higher hazard of leaving the sample, typically because of bankruptcy. In fact, that risk may be substantial because survival is lowest for the very small firms (Davis, Haltiwanger & Schuh. 1996) dominating our estimation sample. We have partly controlled for that by using a potentially more robust growth measure (see Annex 3). Yet, more advanced techniques may use IV strategies such as unexpected deaths (compare Choi, Gold-schlag, Haltiwanger & Kim. 2019). Potentially remaining endogeneity issues are a major reason for interpreting our findings more as correlation rather than causal effects.}. For a solution, We rely on econometric models that control for survivorship bias and the endogeneity of VC financing ! see : \citet{colombo2010growth} \textit{More importantly, as a direct way to control for a possible survivorship bias, we adapt a typical Heckman two-step procedure commonly used in empirical studies on firm growth dynamics (e.g. Evans, 1987; Dunne and Hughes,1994; Lotti et al., 2007) to our specific framework. In particular, we first estimated a probit model on firm exit in the 2000–2003 period conditional on survival up to the end of 1999, again based on the RITA 2000 sample. The independent variables of this sample selection model include founders' human capital variables, receipt of a VC investment before 2000,2 firm-specific characteristics (e.g. firm size and age in 1999), and other controls. Based on these estimates, we computed the inverse Mill's ratio offirm exit for the 439 firms included in the sample (i.e. all 2004 RITA firms with the exception offirms that came into existence after 2000). This ratio was then inserted as a control for survivorship bias in the growth equation. This additional variable controls for the unobserved heterogeneity that affects both a firm's probability of being sampled in 2004 and its growth, allowing more consistent estimates of the parameters of the growth equation.}\newline

- (7) avoid retrospective research falls (cf. Delmar 2006)\newline

- (8) Endogeneity issues (Colombo and Grilli 2005, 2010) --> In econometrics, endogeneity broadly refers to situations in which an explanatory variable is correlated with the error term. ... The problem of endogeneity is unfortunately, often times ignored by researchers conducting non-experimental research and doing so precludes making policy recommendations. https://fr.wikipedia.org/wiki/Endog%C3%A9n%C3%A9it%C3%A9 -> Solution = adoption of a two-step estimating procedure inspired by the “endogenous treatment effect” literature (Heckman, 1990; Vella and Verbeek, 1999; Winship and Morgan, 1999; Greene, 2003. See also Engel, 2002; Colombo and Grilli, 2005a).\newline

- (8.1) ommitied or censored variables. For example in \citep{beckman2007early} \textit{•	Censoring (quand on prend une entreprise, et qu’elle lève de l’argent après mars 2020 → Doit on couper la boite ? non.) “Usually, assume censoring is non-informative”.}\newline/ Furthermore, to control for losts observations or missing data, we can do as in \citet{taylor2006superman} : \textit{To control for observations lost through missing data on creators or publishers, we computed an inverse Mills ratio from a logit model predicting complete data and included it in the model (Lee, 1983).}
Also, in \citet{andries2014small}, they use Heckman 1979 model, and to slve data censoring problem, they introduce  a sample selection model as the performance equation contain a variable that controls for the correlation between tje likelihood to innovate and the innovation performance.

- (8.2) avoid selection biais (cf. Delmar 2006). This problem (coming wth a solution called \textit{matching procedure} is well explained by \citep{engel2007firm} in section 4.2). we have no andomized experiment here, so we have a selection biais well known by microeconomicists --> \textit{"it is virtually impossible to find exact (i.e. “twin”) pairs venture funded an non venture funded firms".} -> Solution = establish a propensity score aka nearest-neighbor Matching Method. This can be solve using Heckman 1979 (correct self selection) or the inverse Mills ratio ; which is also useful to control for missing values.\\

- (8.3) avoid biais de simultaneité\newline

- (8.4) avoid measurement error\newline

- (8.5) avoid common method variance. See : https://www.youtube.com/watch?v=NqJxl5EezDc. \newline

- (9) avoid hindsight biais / memory decay (biais rétrospectif consiste en une erreur de jugement cognitif désignant la tendance qu'ont les personnes à surestimer rétrospectivement le fait que les événements auraient pu être anticipés moyennant davantage de prévoyance ou de clairvoyance)\newline

- (10) avoid reverse causality : This point of reverse CAUSALITY IS MAJOR : endogeneous problem. As such, \citet{grillitsch2020does} said that their research is more about correlation rather than causation. \textit{Specifically, firms which have experienced poor growth in the past may be induced to look out for new or missing skills. The opposite reaction may also be conceivable because past poor growth may make them more passive in terms of hiring. Either way, there may be endoge-neity issues resulting from simultaneity, which we may have accounted for only partly through fixed-effect approaches and a careful selection of controls.} Thus a solution to avoid this is : from \citep{visintin2014founding} \textit{to avoid reverse causality (for example, the team composition may change due to growth), the design of our analysis is cross-sectional and all the independent variables are measured at the year of establishment (t0) whereas growth was assessed at the end of the third year after the start up (t3).}.\\

- (11) Sample procedure
- control for age as in \citep{cooper1994initial}, while focusing on specific techno BUT diverse industry
- From \citep{de2010interrelationships} : the homogeneity of the sample in terms of age, size and strategic focus allows minimisation of extraneous variation. (they pull out from their sample non-innovative firms to keep only innoative outputs). Regarding the homogeneity of the sample, \citet{grillitsch2020does} mentioned that : \textit{By including only young firms, we effec-tively reduce the risk of conflating structural disadvantages, e.g. in recruiting qualified staff, with causal effects. For example, if we included established firms in our sample, positive growth effects of older firms might have been overestimated if those firms also managed to attract more qualified hu-man capital. As all firms in the sample were founded during the sample period, we reduce the risk of systematic biases.}
- As in \citet{reese2020should}, we used a Heckman selection model to control for poten-tial selection effects. Results, including the inverse Mills ratio,
- As in Baron 1995 (Iron cage) we can provide argument that our sample is fine. \textit{By focusing on firms in a single region and economic sector, it holds constant relevant labor-market and environmental conditions, as well as some institutional influences thought to shape structures. Within the Silicon Valley region, we sought industries in which sufficient numbers of comparable firms exist to enable quantitative comparisons. Accordingly, we concentrated on firms engaged in computer hardware and/ or software, telecommunications (including networking equipment), medical and biological technologies, and semiconductors}
- As in Chandler (\citep{chandler2005antecedents}, we can control for representativity with : \textit{Because the response rate was relatively low, we checked to see if our 21/100 sample was representative of the population. We collected information from a random sample of 50 nonresponding firms about the number of founding team members, firm size in number of employees, sales level, and SIC codes. There were no significant differences between the responding sample and the nonresponding sample in number of employees, sales levels, or industry representation. Thus, we believe our sample to be representative of the population.})
- As mentioned by \citet{grillitsch2020does}, \textit{sample selection is a particular issue in many entrepreneurship studies because population data or at least representative samples often do not exist.}

- (12) Classifications of competencies : - As in \citet{reese2020should}, Cohen's kappa of 0.70, which represents substantial agreement (Landis & Koch, 1977) + 2 others mentioned in the study.

- (13) Flexible specificatons about the time dependence (very useful for our dataset). Mentioned by \citet{grillitsch2020does}.

\section{Results}

####### FIGURE / TABLE OF CORRELATIONS / CAUSALITY ########

A nice way of presenting the results : see 1/ \citet{taylor2006superman} = well explained


\section{Contribution and limitations}

Use other indicators of performance : cf. Chloe pitch.

Limitations : same as in \citep{ko2018signaling} \textit{First, we focus on a specific type of venture (i.e., high-growth ventures) in a specific industry (i.e., internet advertising). While there are benefits to focusing on a single industry, we are not able to generalize our findings to other industries, such as those requiring higher levels of initial technology (e.g., biotechnology) or intensive upfront capital (e.g., telecommunication). In addition, the scope of this study is limited to growth-oriented ventures since these are most likely to solicit external financing (Zider, 1998). Thus, the findings in this study may not be applicable to other types of ventures, such as hobby ventures or traditional small businesses, which generally do not pursue similar funding routes. Second, the amount of funding new ventures receive not only indicates the signaling effect of ventures' competence and legitimacy but also reflects their goals for external funding. Not all founders will seek maximal external financing since this implies diluting the founders' equity stake (Zider, 1998). While we attempted to address ventures' heterogeneous goals via robustness tests of unobservable heterogeneity, we acknowledge that there may be other factors at play in understanding the amount of funding received.}

Since\citet{beckman2007early}, we can adapt and talk about ENTIRE TEAMS + building ON 2 background affiliations --> \textit{Empirically, the study has two contributions. First, background affiliation has been rarely examined in the literature. Second, we are able to decompose turnover into entrances, founder exits, and team exits.}, Also, as in this paper, 3 limites : comme dans Beckman : 1/ Data completedness (media-shy) 2/ no measure of team process or personnality and power within the firm 3/ data specific in time and space (may be affected by internet boom).

Since \citet{grillitsch2020does}, one limitation we can have is that life cycle theories applied to HC and growth of venture does not answer the question : why firm hire. \textit{These theories however implicitly exclude the question of why firms hire and therefore do not unpack the potentially complex intertemporal rela-tionships between growth and the recruitment of skills. Our viewpoint is consistent with supply-side views: e.g. Penrosian growth theory would contend that firms with excess skills will expand (poten-tially into other markets), where these skills can be fruitfully applied. Demand-side views would sug-gest that firms meeting excess demands will recruit new skills. Here the cause-and-effect relationship may be turned around. While our data does not allow us to identify the firms' motives for recruiting new skills, the various robustness checks, in particular those relating to the lag structures, do suggest that supply-side explanations, where new skills cause growth, are playing a role - even if it is not nec-essarily an exclusive one. Nonetheless, exploring the causal micro-mechanisms relating growth to skill recruitment would clearly be a logical next step to further develop the ideas set out in this paper.}

\section{Policy implications}

As in \citet{chandler2005antecedents}, which based its theoretical framework on Boeker 1997, the dynamism in turnover of employees is seen as an ADAPTATIVE MECHANISM.

See Mason and brown 2014 - Inside the high-tech black box: A critique of technology entrepreneurship policy
Author links open overlay panel --> The paper offers suggestions for how policy could be recalibrated to better reflect the requirements of local entrepreneurial actors.

Since \citet{andries2014small} \textit{"the historical focus on the entrepreneur/CEO which was broadened more recently to the study of entrepreneurial teams does not fully capture small firms’ innovative potential}.


%References section - bibliography
\clearpage
\bibliographystyle{plainnat}
\bibliography{biblio}

\end{document}

latexmk -pvc -pdf -xelatex -interaction=nonstopmode foo.tex
